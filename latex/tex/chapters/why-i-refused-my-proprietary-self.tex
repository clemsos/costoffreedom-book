\subsection{Why I Refused My Proprietary
Self}\label{why-i-refused-my-proprietary-self}

\begin{quote}
\href{../appendix/attributions.html\#adam-hyde}{Adam Hyde}
\end{quote}

I find myself, after all this time immersed in free culture, amazed at
my holding on to some form of the proprietary way of thinking. Sometimes
I have found myself consciously going quite a long way down that path
before I stop myself and almost forcibly ask myself ``hey! What are you
doing?''

Some time ago I started a methodology called the Book Sprint. It's a way
to facilitate the production of books in 3-5 days with a group of 6-12
people (or so). It took a long time to hammer out this method. Much
financial, personal, and emotional pain to keep going down a road that
nobody, including myself, really understood terribly well. Was it really
possible to make it work? Well, it took about 4 years of hammering on
this methodology, making plenty of mistakes, before I could actually
think about it as a methodology. Before I could actually wield it with
some form of embryonic artistry, see it in action, build upon it,
improve it, teach it to others.

4 years is a long time. It felt like a long time. Truth is, I don't
really know why I didn't give up, and my stubbornness is something that
kind of shocks me, looking back.

Suddenly I could see the prospect of a sustainable lifestyle emerging.
How would I make it happen and protect it? I had this horrible feeling
that I was not good enough at scaling the project and some big ugly org
with heaps of cash would scoop in and `steal it'. I guess I meant they
would swoop in and copy it. The danger of a ripped-off dream caught me
off-guard and I went down the road of lawyers and trade mark protection
for Book Sprints. This was my first step towards owning the methodology.

I look back at that now and I'm kind of amazed I went down that path as
far as I did. I didn't actually follow through with trade-marking. The
lawyer told me it was going to cost more and more, and it gave me time
to wake myself up. What was I doing? The fear of losing my creation led
me down a blinkered ``IP'' way out of line with my personal politics. It
brought me awareness to peel off the layers of proprietary living that
transpires our skins.

The process is a process of painful personal growth. Sharing my
experience with hardened free culture practitioners, I've met quick nods
of agreement. Only the idealist newcomers look puzzled at my apparent
failure: I'm not a true believer. There is no purity on the path to
freedom. Walking through the shameful path of not meeting the high bar
we've set ourselves to avoid proprietary life, I keep learning about how
deeply embedded it is in our daily lives. I keep examining it and it
keeps surprising me. I keep discarding it. There's still a long way to
go.

Edited by hellekin, 3 Nov 2015, Pourrières, France.
