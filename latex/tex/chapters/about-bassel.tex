\subsection{About Bassel}\label{about-bassel}

\begin{quote}
\href{../appendix/attributions.html\#patrick-w-deegan}{Patrick W.
Deegan}
\end{quote}

Bassel Khartabil (Arabic: باسل خرطبيل‎) also known as Bassel Safadi
(Arabic: باسل صفدي‎) is a software developer and community builder, an
advocate for internet freedom, and most recently, and perhaps most
personally, a supporter of free-access and liberty in Syria.

Bassel's work in Syria joined his numerous other international projects
together into a unified and focused opus. These earlier works included
worldwide work with Mozilla Firefox, Wikipedia, Openclipart,
Fabricatorz, and Sharism, as well as being an initiator and key member
of the Creative Commons Syria release. Khartabil also developed the
novel web framework known as Aiki as a part of his own collaborative
research company Aiki Lab. For its own part, Aiki codified many aspects
of Bassel's own personality: surprisingly user-friendly while being
technically sophisticated, Aiki is a web developer's concept of poetic
code in its powerful simplicity.

Taken together, his most recent work --- New Palmyra --- sought to
capture in a similar spirit of public openness one of the hallmarks of
human civilization. New Palmyra presents a digital archive in rendered
3D of the ancient site of Palmyra. At almost every level, from process
to function, and from code to metaphor, this project is as an almost
perfect stand-in for Bassel himself. And perhaps it also summarizes in
form and idea the fact that Bassel is presently not here.

Since mid-march of 2012 Bassel has been a prisoner of the Assad regime
in Syria. No longer a country satisfied with the politics of As-If,
Bassel was long an active part of asking for the very best of Syrians,
for themselves and for the world. And for this, like so many of his
fellow countrymen and women, he was imprisoned. But Bassel knows that a
community is a powerful thing --- it retains and rebuilds, it preserves
and presses forward, and above all it never forgets its own.
