\subsection{Queering}\label{queering}

\begin{quote}
\href{../appendix/attributions.html\#natacha-roussel}{Natacha Roussel}
\end{quote}

From 23 to 24 May, \href{http://femhack.org/}{femhack} organized an
international hackathon in the loving memory of Sabeen Mahmud, getting
together amongst a large number of feminist hackerspaces locally and
around the world.

Sabeen Mahmud was a Pakistani activist fighting for human rights in
Pakistan. She was the co-founder and director of the second floor (T2F),
a cafe in Karachi. She also had been the president of Karachi's branch
of deTiE (The Indus Entrepreneurs), a not-for-profit organisation
dedicated to promoting entrepreneurial spirit. On 24 April 2014, she was
shot down by unidentified gunmen while coming back from the seminar she
had just hosted at T2F, examining issues and triggering awareness about
people who had disappeared in Baluchistan, a province of northern
Pakistan.

A year after her death, we had the desire to express our solidarity
online and off-line, as a network of feminist spaces for resistance,
being transnationals and postcolonialists. Furthermore, this event
allowed us to more clearly define our network of solidarity. We do have
a shared discourse, and we also work to appropriate technological space
to the benefit of our communities. We feel we are engaged in a larger
process that fundamentally nurtures our small community-based
structures. Most of us consider we are in a sphere of action that
overcomes the deconstruction process needed to get out of a proprietary
way of life. We put forward alternative ways of life and solidarity
networks. Our next concern is to secure our existing structures: this is
not an easy process, as fragility is also a definitive asset allowing
for sensitivity and understanding. However, while numerous, our
structures lack the sufficient visibility that would allow better
protection, and consequently it keeps being difficult to identify
everyone.

This day was the occasion of an encounter that has enabled us to
identify one another better: since then; we continue to exchange
messages on a dedicated mailing list that helps us to know each other
better. However, it still is very difficult to completely identify each
other in the varied materiality of our different commitments. Since that
day, the more than 30 structures in which we are participating have
developed a series of approaches to the issues, going from Wikipedia
editathons to augment feminist content on Wikipedia, to Women in
Surveillance meetups, citizen-sensing endeavors, or small exchange and
programming groups. However, despite the persistent relations that we
are creating and the commonality of our interests and attitudes, it
remains a complex challenge to understand and assess the personality of
each of us in an always-transient state of being, as people are involved
in projects with different levels of risks.

All continents were represented during that event, but the most numerous
were situated in Latin America, maybe because of the beauty of a
language practice that has invented a written transgender form; for
example: ``somos guapxs`` is the transgender form of ``we are
beautiful.''
