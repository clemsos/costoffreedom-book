\subsection{Palmyra 3D, Premonition Vision of
Bassel}\label{palmyra-3d-premonition-vision-of-bassel}

\begin{quote}
\hyperlink{faraj-rifait}{Faraj Rifait}
\end{quote}

Born from a Palestinian father, writer, and Syrian mother, Professor of
Piano, Bassel lived in an environment open to the world and remote from
any conservatism. From his early childhood, reading was a refuge for the
only child of the family. While children of his age were playing with
toy cars, Bassel had already gone beyond the comic and was devouring
books about the ancient history of the Middle East and Greek mythology.

Living in France, while Bassel grew up, one day he surprised me by
speaking to me in English with a very rich vocabulary. He was only 10
years old. I asked him if he had learned English at his school in the
Palestinian camp in Damascus. He smiled slyly and replied that it was
through his father's computer, using a CD that he had learned English.

At 11, he had his own computer, donated by his mother for his birthday.
I was expecting that he would play computer games, but I was wrong.
Bassel showed me his computer programs in C language and translations
into English and Arabic of some historical books. Thus, he helped his
father in his research and writing his books on history.

I was surprised to see him acquire advanced technical skills for a 12 or
13-year-old boy, but Bassel told me that his uncle Osama, a computer
expert at the time, helped him to develop his natural gifts. Two years
later, Osama assured me that now it was he who asked advice from Bassel.
As a teenager, Bassel appeared to me very passionate when he resolved
computer programming, sometimes very complex projects. It seemed that
Bassel was traveling through the computer world and the history of his
country in a very special and multidimensional way.

It is from this double passion for history and computer programming that
Bassel began working on technical projects like the creation of a web
site on the discovery of the archaeological treasures of Syria. He was
barely twenty years when he begun the Palmyra project in 3D, in close
collaboration with Khaled al-Assa'ad, the great expert of Palmyra
history, who was beheaded by Daech in 2015.

Bassel has a great intuition as if prescient, that may explain why he
initiated this beautiful and ambitious project to safeguard the memory
of this outstanding universal site. Bassel wishes that everyone could
reinforce and contribute to embellish this multidimensional work in
these troubled days until his release from jail\ldots{}
