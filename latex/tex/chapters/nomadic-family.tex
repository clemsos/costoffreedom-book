\subsection{Nomadic Family}\label{nomadic-family}

\begin{quote}
\hyperlink{natacha-roussel}{Natacha Roussel}
\end{quote}

The problem of the costs within the schizoid logic of our times concerns
mostly potestas, the quantitative, not potentia, or incorporeal
intensities. --Rosi Braidotti, Nomadic Theory

The question of costs often translates into issues of scale and
scalability that are dominant in technological societies. The ``scale
solutionism'' starts from the desire to solve cost problems and ends in
hyper-control, restriction, dissociation and finally disaster conducted
by non-aware necropolitics, where the politics of death systematically
takes over the politics of life (Mbembe 2003), increasing the costs of
freedom. In such instances, when the state of power constantly refers to
a state of exception in order to overcome the rule of preservation and
the social limit, Achille Mbembe explains that it seems figures of
sovereignty develop a general concern that is not the preservation of
the commons and liveliness, but the spreading of death and the material
destruction of bodies and populations: Bassel Khartabil is,
unfortunately, a direct victim.

In this context, it is impossible to address the problem of costs
without transforming our relation to the existing system. Always
confronted with an impossible dilemma of sustainability, we need to
envision different ways to face this situation. While costs are most
often evaluated as a quantifiable asset, this quantification is mainly
calculated in regards to an actual neo-liberal vision of individual self
and proprietary systems. It seems crucial to envision different avenues
to overcome the cost issues, and define new criteria of cost evaluation
that could lead to re-thinking the free production processes in a
different organization scheme, resulting in the main question: we should
ask ourselves if the costs of freedom cannot be addressed as a
qualitative process rather than a quantitative one.

Practically, to enforce such a process, only the diversity of networks
can help secure our individual endeavors; therefore, the re-evaluation
of the cost of freedom should start from the premises of community and
collective approaches to production and network realization, which
support non-proprietary production and distribution of information.
Resulting from the contestation of the need to encompass our work in
active F/LOSS and open source developments, is the necessity to situate
our social connection and embodiment leading to new contexts for such a
production. Starting from an assertion of the actual situation, we are
looking at ways to think complexly with regards to freedom issues, and
explore how to co-synchronise so that the relation that feeds our
networks can exist despite actual power issues.

A Foucauldian view of the actual context would present, coextensively to
the rise of power structures, the formulation of scientific discourse as
the cause of actual costly body politics. While modernity has attached
its project to a rational view of the world based on a clear mind-body
split that is exponentially growing along with technological
development, this disunion nurtures the dissociative powers of
capitalism. Despite all efforts to enforce a discourse promoting
technology as a substitute for human relation, it is, however, certain
that the posthuman does not map to the network, and more specifically it
appears that the proposed agenda of dematerialisation and autonomous
artificial intelligent networks is a fantasmagorical construction
(Hayles 2001). Therefore, it is from a holistic perspective that the
observation of the actual complexity needs to be undertaken. In the
context of a huge up-scaling of human presence on earth and the growth
of social control apparatus, can an examination of relational complexity
bring us towards social sustainability, and what would be the sensitive
approach that could ground an exchange system, and lead it towards a
sustainable expansion?

A holistic setup would allow us to spare ourselves by leaving the costs
for freedom at the expense of the potestas while reacting in diverse and
unstructured networks, and at a molecular level to reach full potentia.
We are looking for ways to confront necropolitics and trigger
liveliness; in this context liveliness is to be thought as a spiritual
process that further constitutes the grounds for a different politics.
Indeed, a different approach to politics needs to be rooted in the life
of the spirit that is not afraid of death, and instead of looking for
substitutes and technological prosthesis, it fully assumes death as a
constituent of human relation and organisation while it looks beyond the
unitary vision of the self, to molecular transformations as a way to
synchronize to the world in a deeply transformative process (Braidotti
2011).

In response to this statement, several issues need to be addressed that
would further ground the development of our community processes, based
on a long history and knowledge of existing knowledge. Some affordances
might lead to explore different relational setups that would help to
transpose the question of costs.

TRANSMISSION: While power relations build over cycles of crisis, they
seem to destroy reference points and instrumentalize history to the
service of immediate power relations. Indeed, it is clear that
technological breakthroughs importantly transform relational processes,
but contrary to what we once have thought, they do not expose the
processes of power. On another hand, critical discourses, tools and
concepts are developed through time, and they often are sourced from
fragile social structures, either isolated individuals or community
structures. As a consequence of this fragility, they most often
repeatedly deal with recurrent issues, while transmission lines are
broken, they each time face the need to develop a discourse and
solutions. It is important to intervene at community scale in the
process of transmission to create community genealogies and a history of
community movement through time. This would allow us to keep those
principles active during technological transitions. One of the
possibilities is to expose current technological communities to existing
social science and allow for transdisciplinarity and politicization of
the discourse. The project of hackerspaces workshops, for example,
inscribes itself into a transactional process of transmission through a
collective community context.

BIOPOWER: As it appears that sovereignty stands as a condition of
control, the question of the unicity of self, is again a transient issue
persisting across time and through technologies. Variations of intensity
characterize the thinking subject and are mostly characterized at its
boundaries; those variations set a relational process independent from
the view of a holistic body. They in principle go far further than the
limits of human species in setting the potential of transformation into
a process of becoming. According to Rosi Braidotti, this
denaturalization process is one of the effects of technological progress
in fields such as biogenetics where we integrate different species in an
inter-evolutionary process.

TRANSFORMATION: After a consciousness-rising process triggered by the
awareness of a state of dismay, it could be timely to consider, observe
and acknowledge a trans-species potential for knowledge diversity
leading to social sustainability. This process can be thought as both
individual and collective, implying both personal mutation, and through
collective support, a larger transformational process. Being in the
instant and acting from this perspective, and responding to the trigger
of the momentum is a way to reach the acknowledgment of the possibility
of instantaneous transformation. Variations of codes, genres and
modalities of expression of the idea see transposition as a possible
solution for genetic transmutation and exchange.
