\subsection{Supporting Bassel}\label{supporting-bassel}

\begin{quote}
\href{../appendix/attributions.html\#ethan-zuckerman}{Ethan Zuckerman}
\end{quote}

Bassel Khartabil has been imprisoned in Syria since 2012 for the vague
``crime'' of ``harming state security''. Near as anyone can tell, his
crime was in being an advocate for the use of the Internet as a platform
for free speech. Through his promotion of open source software, his
leadership of the Syrian Creative Commons community, and his work
building innovative new publishing platforms, Bassel worked to connect
Syria with the rest of the world, and to ensure that all Syrians --
supporters of Assad and opponents - could make their voices heard
online, even if they could not express themselves in physical space.

Our work on Civic Media at the MIT Media Lab stems from the idea that
making media is a way of making change in the world. Bassel's work is in
the best spirit of Civic Media, working to connect contemporary Syrians
to global conversations while preserving Syria's rich history and
culture. Before his unjust incarceration, Bassel was working to build a
3D model of the ancient city of Palmyra, much of which has been
destroyed by ISIS fighters in the past few months. At this tragic moment
in history, Syria is losing its physical history to religious fanatics
while persecuting the people who could be building their digital future.
