\subsection{Beyond Capitalism}\label{beyond-capitalism}

\begin{quote}
\href{../appendix/attributions.html\#hellekin}{hellekin}
\end{quote}

The future was to be excellent. Thanks to the endless progress of human
knowledge, technology would deliver the right solution at the right
time. As industrial powers scaled up, though, and hacked their way out
of diminishing returns with brute force, the picture of a bright future
turned out to be as naive and grotesque as the vision of the year 2000
as seen from 1902.

Modernity is totalitarian. Following Descartes' proclamation of the
prevalence of the mind over matter, modern science engaged in a process
of stripping away uncertainty and contradiction. The world of the
mechanical clock was thoroughly explained, controlled, and made to serve
mankind, in accordance with the Biblical injunction of breeding and
multiplying, and using the God-given resources of the Earth. But the
world is not complicated: it's complex, and contradiction is built-in.

Capitalism was a fantastic booster that propelled us from candle light
to LED, from parchment to digital computer, from horse carriage to
spacecraft. Its premises, though, require endless growth, and some time
would pass before we could replicate our own spaceship Earth. As it
attained global operational scale, capitalism was panting like a hamster
on its wheel ready for a heart attack. The myth of progress was on
artificial respiration. The capitalist system now reached capacity and
still requires new markets, better outcomes, more efficient ways to suck
fossil and mineral resources off the ground. The system is ticking
seamlessly: grab a piece of primary rainforest, cut down the trees for
construction and furniture, plant soy to feed millions of pigs on
thousands of farms, then when the soil is sucked dead 5 years later,
mine for minerals and frack for oil shale.

We would already need to harvest the resources of four planets like
Earth to keep up with the pace at which the global industrial war
machine exploits and decays our environment. But we barely can send
robots to Mars, so this option is off. We could wait for the next
super-technology-that-will-save-us-all, but as Jevons observed, any
technological progress increases the efficiency of resource use,
consumption of resources rise as more demand is met. If a new engine can
be made more cheaply, it will sell more, and the net result will be a
faster and stronger pressure on resources. Even if such super-technology
could potentially appear, it remains a big IF, and would it come in time
for us to reverse the damage already done to the fragile conditions that
maintain the Earth livable for our species?

An obvious course of action would be to stop running and relinquish a
bit of comfort to bring about the possibility of our survival. This
solution, though, requires the end of growth, which fundamentally
contradicts the extraction system that fueled the technological boom in
the last two centuries. Given the importance, in terms of scale, of the
problem at hand, the possibility of a peaceful solution remains both
remote and indispensable. Other paths can only amplify the crisis and
lead to catastrophe.

Thought, here, has reached its limits: only action remains possible.
Mindful, ethical, and compassionate action. Loving, caring, and sensible
action may unlock the true potential of a successful humanity, and
freedom, yet undefined, remains a golden key.
