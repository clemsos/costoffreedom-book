\subsection{Image, Identity, Attribution,
Authorship}\label{image-identity-attribution-authorship}

\begin{quote}
\href{../appendix/attributions.html\#christopher-adams}{Christopher
Adams}
\end{quote}

We can say that a photographer owns her images, in the same way that an
author owns her words. (The shorthand for this ownership is copyright.)

However, we should not conclude that the photographer has rights to her
subjects, legally or morally, in the same way that an author has rights
to her ideas. The reason that a photographer cannot make claims upon her
subjects is that her work crosses that boundary between persons, and
persons have their own rights which must be considered.

We are not so ignorant as to say that a photograph will steal our soul,
and yet we are dimly aware of a danger in pictures of our faces or
bodies, as if something can be taken from us, and get away, to who knows
what end. We know there could be a ``cost'' to each photograph that is
taken.

Photography did not always enjoy the protections of copyright. The
argument went that manipulating a machine (in this case, a camera), did
not count as a creative act.

Eventually, the question of whether making a photograph rose to the
level of authorship was settled by the courts, in the affirmative.
Photographers are the authors of their creations and thus own the
copyright.

That photographs are protected by copyright also means that a
photographer is free to release her work under a free license that
allows others to use, reproduce, modify, and learn from her creations.
The minimal requirement to re-use a freely licensed photograph is the
simple gesture of giving credit or attribution to the photographer, in
the manner she specifies.

Free licenses apply to a photographer's rights as an author, and your
rights as a user. However, they are silent on the legal and moral rights
of the subjects of our photographs, which we might understand as the
right of publicity.

In order to secure this additional right, the photographer must ask
something of her subject.

The subject must consent not only to the photographer's use of his
image, and to others' re-use and modifications of his image; he must
also permit his name and identity to be associated with his image. That
is the ``cost'' of the ``freedom'' of his picture. He lets a fragment of
his soul escape out into the world, forever.
