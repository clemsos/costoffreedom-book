\subsection{Introduction}\label{introduction}

Freedom comes with many costs, not least responsibility. Social,
psychological, financial, bodily, emotional: known and unknown costs,
often to bystanders, make any strategy to gain and protect freedom an
ambiguous quest. Sometimes it isn't clear what freedom means. Many
people use and produce bits of free knowledge, but any serious attempt
quickly runs into tremendous barriers, in every field. Participants
receive unequal welcome due to gender, language, cultural or economic
differences. Occasionally, the production of intangible assets may
intersect with broader historical movements, redefining their meanings
and exposing their participants to unlimited costs.

Considering the costs borne by millions to obtain, for example, freedom
from slavery or freedom to vote, free knowledge movements seem rather
safe and straightforward. By contrast, to consider the costs of free
culture, free software or open scientific research may look adventurous,
or perhaps just presumptuous. But this is what we will attempt to do,
with appropriate humility. This book wants to discuss how free knowledge
movements are built and the real costs attached to them. Activists,
artists, designers, developers, researchers, and writers involved with
free knowledge movements have worked together to see further than the
fog of our news feeds and produce some sense from our different
experiences.

This book is born in an attempt to free Bassel Khartabil, loved and
celebrated Internet volunteer detained in Syria since 15 March 2012. His
name has been deleted from the Adra Prison's register where he was
detained, on 3 October 2015. We have not received any information about
his current status or whereabouts since. The introductory part of this
book called \href{collective-memory/index.html}{Collective Memory} gives
voice to his friends and family that have been urging for his release
and want him back in his normal life and freedom, immediately.

Seeing Bassel paying a high price for his participation in free culture,
many of us have started to reflect on our own fates, actions, and
choices. Why are we here today? What have we chosen? What have we given
up in this process of sometimes extreme belief? The second part,
\href{opening\%3Afreedom/index.html}{OPENING: FREEDOM}, is a
recollection of personal, sometimes contradictory reflections and views
about the experience of working within free culture for some years. The
diversity of contributions express the many directions that have been
taken to act.

The third part called
\href{architectonics-of-power/index.html}{ARCHITECTONICS OF POWER} takes
a step back to look at how we, as a society, deal or fail to deal with
the different barriers that stand in our ways towards freedom. Different
authors analyze the contradictions of their choices and daily activities
with larger objectives and lifestyles associated with free culture. The
variety of professions and situations of the contributors offer an
illustration stained with multiple tones.

Finally, the fourth part \href{affordances/index.html}{AFFORDANCES}
offers a reflection on theories and successful practices of free
culture. It offers different perspectives on the nature, structure,
motivations and limitations of existing levers towards liberation, not
only legal and technological but also social and cultural.

Once marginal, free culture is today on the edge of becoming part of the
new normal thanks to the Internet while being threatened in its
fundamentals by its own success. The many contributions in this book
offer a unique snapshot of its dreads and interrogations, and a
tentative program for the reader to reflect on the future of freedom in
our times.
