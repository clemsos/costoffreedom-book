\subsection{Collective Validation}\label{collective-validation}

\begin{quote}
\href{../appendix/attributions.html\#ginger-coons}{ginger coons}
\end{quote}

A friend of mine---a multimedia artist and a community organizer---once
referred to taking a job at a commercial software company as ``becoming
a civilian,'' as something that might be a little more relaxing, a
little lower pressure than what she was used to. What she meant was that
by just taking a job, a normal job, with normal expectations, she was
opting to get out of the public eye for a while, to stop doing work that
could be seen, judged, and assessed by the whole world.

I know the feeling. In all of the free cultural work I've done over the
last six years, one of my most pervasive anxieties has been the feeling
that I work in public, that everyone is always looking over my
shoulder---or could be if they wanted to. It's a difficult feeling to
come to terms with, even if it's based on one of the most potent and
valuable principles of free culture: transparency.

For the last five years, I've worked on a project called Libre Graphics
magazine. The point of it---the whole point, to my mind---is to show off
just how good graphic design and art done with Free/Libre and Open
Source software, standards, licenses, and methods can be. It's the whole
package, and the whole package includes a kind of extreme transparency.
For five years, my collaborators and I have stuck all of our working
files into a public version control system. For five years, we've opened
ourselves up to scrutiny and criticism not just when we put out an
issue, but before, during our development process. As with free
software, one of our goals has been to release early and often, to make
our work public so it can become better. We don't hide our production
files and then release when they're perfect. It's nerve wracking to work
in public like that, even if most people aren't digging through the git
repository and looking at our working files.

It's nerve wracking and sometimes even scary to open yourself up to the
potential for scrutiny all the time. But it's still valuable. If the
point of Libre Graphics magazine has been to show that F/LOSS and free
cultural principles apply outside of software, then that potential for
scrutiny has been essential. If the point is to show that designers can
do high-quality work with F/LOSS, then the potential for scrutiny is
also the opportunity for someone who's feeling uncertain about even
trying something new to come along and see how we did it. The publicness
is a chance for others to follow in our footsteps and to use our
mistakes to do things better in the future, or to skip over some of the
tough bits. That publicness, in short, is worth something.

But on a personal level, it's still nerve wracking. It can be
frustrating to pour your heart---and worse, your time and your
effort---into work that's totally voluntary, with almost all rewards
intrinsic. The freedom to create, to put something out, to experiment,
to try, is also the freedom to be ignored, to be undervalued, and at
worst, to be bashed or harassed for your efforts. You can't rely on the
positive feedback from others to keep you going. You have to enjoy and
value what you're doing for itself. And if you succeed, if what you make
is something that others find valuable, that breeds the expectation that
you'll continue, even if the odds get long. It can feel as if you've
gone from being ignored to being taken for granted.

And then there's the old aphorism about free software being free like
speech, not like beer---free as in freedom, not as in money. But the
best variation I've seen is free like a puppy: if you adopt it, you
become responsible for it. You care for it. By taking up the chance to
do something, you take up the responsibility to keep doing it, often at
personal expense. And it can get pretty expensive. It can be expensive
in the normal ways, what we typically mean when we use the word
``expense,'' but more importantly, it can become emotionally expensive.

Celebrities and politicians get paid commensurate with the expectation
that their work will be judged by the public. People working with F/LOSS
and free culture generally don't. We do it because we love it, or at the
very least, because we believe in it. And we believe in currencies other
than money, too. We often believe that the work is its own pay and that
it doesn't take money to be worth the occasional frustration of having
others be demanding of our time and effort. But the costs are real.

One of the other costs of freedom---of the transparency I value so
highly in free culture---is feeling as if you're never allowed to get
something right. When your work is done in public and when its success
is often a matter of public opinion, it's easy to feel as if every
decision you make has the potential to be second-guessed. For every
little snippet of positive feedback, for every bit of evidence you get
that your work has made a difference, there's a horde of people who are
happy to tell you every little thing you've done wrong. That happens
when you work in public, and it can be powerfully demoralizing.

Because it's worse for ongoing projects, it can make you wish you hadn't
chosen to aim for continuity and accountability. A one-off, something
you make because you feel like it, throw out into the world, and then
don't plan to invest in over the long-term, doesn't need the ongoing
commitment, the continued desire to engage. When you explicitly choose
to do something in the long term, to commit to a project that lasts and
grows, when you commit to becoming a fixture, the drip-erosion that
comes from the second-guessing can be enough to scour away the desire
that originally drove the project.

When we build free cultural projects, we try to enrich the world. We do
things, not just for our own benefit, but because we think we can do
something good for others. Releasing work under licenses that allow
others to reproduce, to rethink, to remake and to re-release is an
explicit commitment to the commons, and to the idea that we can build on
the work of others, and that others can improve or change our work. When
we undertake to do work in public, we commit to something similar. We
commit to the idea that others can derive value from seeing our process
and that we can grow and improve from having our process intervened in
and commented upon. When we build collaborative projects, we make a
commitment to inclusion, to allowing others to work with us if they
share an interest in the project and willingness to contribute. These
are valuable commitments, driven by a desire to help others and to
enrich the commons. They're important and they matter. These commitments
are the foundation of free software and free culture.

Principles are important. Ideals are important. Sometimes, though, it
feels as if we get crushed under the weight of their downsides. It can
be profoundly demoralizing when it feels as if most of the feedback you
get is negative. And that doesn't need to happen. I long for the day
when we all---even as strangers who only meet when judging each others'
work---think about how much effort, time and personal expense goes into
the things we release. I long for the day when we decide that looking
after other creators and contributors matters, even if we don't really
know each other. I long for the day when there's more positive
affirmation than judgement. And most of all, I long for the day when we
recognize that we're all mostly fighting for the same thing---for
meaningful contributions to the commons, for a way to build culture
together. I so look forward to the day when we can accept not just that
others produce work we can judge, but that the people producing those
works are humans, as fallible and delicate as we are, and that they
deserve not just our feedback, but our praise and encouragement.
