\subsection{Too Poor Not to Care}\label{too-poor-not-to-care}

\begin{quote}
\hyperlink{ben-dablo}{Ben Dablo}
\end{quote}

I am writing about free culture from the perspective of someone who uses
free software and consumes free culture because it's the only thing I
can afford. Although I pay a heavy opportunity cost, the alternative is
supporting proprietary regimes that are actively making the world worse.

At the risk of my future social and economic mobility, I have a
confession to make: I don't have much money. I'm one minor disaster from
being completely reliant on the generosity of others again. Poor is the
common way of putting it, but I try to avoid self-labeling as such since
that would invite further disadvantage. Why I don't have money is a
personally well-trodden topic, but for many reasons I won't discuss,
it's a common state for many people. Despite having little capital in a
society that places so much emphasis on capital, I consider myself
fortunate.

I have the privilege of writing these words using free software on a
(mostly) free operating system. My computer isn't even modern enough to
run a currently supported proprietary OS. Much of my education and
character can be traced back to free culture sources. The novel ``Down
and Out in the Magic Kingdom'', by Cory Doctorow, introduced me to a
world where the alternatives to closed systems could win, where one
could even thrive without the motivation of securing as much private
ownership and IP as possible. It was released under a Creative Commons
non-commercial license that inspired me to write, freed from the
assumption that I must always choose between success and my principled
opposition to proprietary regimes.

That's how I felt over ten years ago as a student, but a decade on, I
wonder if the cost has been worth it. Perhaps I'd be financially secure
if I went with Microsoft products in the developer space instead of the
LAMP stack. Maybe I'd be a successful musician if I had spent my meager
funds on proprietary music production software instead of struggling
with free software packages that were often incomplete by comparison. Of
course, all my unrealized potential could simply be attributed to my own
shortcomings as a developer, musician, and writer. But what about
everyone else, the young people that may someday be asked to choose
between free culture values and success?

I would tell them without regret that I would choose the same path.
Success at the price of one's principles is really an ethical failure
framed the wrong way. In the past such a statement could be interpreted
as melodrama; being forced to pay a small fee to consume old
entertainment media is hardly the most pressing issue, but today the
costs of closed systems and proprietary regimes are plainly manifest on
a global scale. In the human rights space, free software contributors
build tools used by dissidents, activists, and whistleblowers.
Proprietary vendors, when they're not busy adding backdoors to their
software at the behest of governments, largely ignore those groups. In
the environmental space, free knowledge contributors make educational
videos and texts freely available to millions, while traditional
publishers print books on established topics that are bound from birth
for the landfill as next year's edition will supersede this year's.
Duplication of effort on a massive scale to get around someone else's
intellectual property has become yet another unnecessary source of
carbon emissions.

Freedom has many costs. It might even prevent you from ever being
materially wealthy. However, sacrificing our ideals when so much
external to ourselves depends on them is a cost we can no longer afford.
