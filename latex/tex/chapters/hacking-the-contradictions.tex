\subsection{Hacking the
Contradictions}\label{hacking-the-contradictions}

\begin{quote}
\href{../appendix/attributions.html\#stephanie-vidal}{Stéphanie Vidal}
\end{quote}

Contrary to most people's belief, contradictions are an interesting and
powerful tool for the thinking process. They are something we often
encounter in our minds. For some people, contradictions are the ultimate
roadblock, whereas for others they are a just a stage of their
reflection. The latter group, when realizing they are stuck with an
insoluble confrontation, find the strength to make a sidestep in order
to move forward.

With a twist, a bounce, an awkward move, they are able to overtake the
contradiction, to hack it and go beyond.

We are at the edge of a critical moment where aporia are not only
rhetorical, but contradictions are systemic.

Having a close look at the digital media fields, we find those
contradictions at the core issues of contemporary productions such as
art, critical design, literature, code, technology, philosophy,
economy\ldots{}

Most of them are seeking for disruptive technology, social innovation,
philosophical renewal: they can be called a sidestep.

Maybe freedom of thought, to make and move, appears when we are able to
make a sidestep Maybe freedom is just the possibility to take a sidestep
Maybe freedom means the power for the people to stay in control of their
dreams despite the reality of the system Maybe freedom is the way we
ethically manage internal and external contradictions to go beyond.
