\subsection{My Brain on Freedom}\label{my-brain-on-freedom}

\begin{quote}
\hyperlink{mike-linksvayer}{Mike Linksvayer}
\end{quote}

A cost of participation in free knowledge movements is ``stupidity'' --
an assault on intelligence, wisdom, reason, knowledge. The net effect of
free knowledge on intelligence is probably positive, possibly hugely
positive if free knowledge movements succeed in thoroughly commoning the
noosphere, making collaboration and inclusion the dominant paradigm for
all economically valuable knowledge production and distribution. But the
stupidity costs of free knowledge are real and painful, at least to me.
Fortunately the costs, if acknowledged, can be decreased, and doing so
will increase the chances of achieving free knowledge world liberation.

I want to explore briefly how individuals, communities, and society are
affected by various kinds of costs of free knowledge. This is going to
be cursory and incomplete. Very possibly also stupid: my mind has been
infected by free knowledge for about 25 years.

Commitment makes us morally stupid, lazy, and unconvincing. Claiming
that knowledge freedom is a moral issue is not a valid moral argument,
but merely an unsupported claim which ought be embarrassing if not
immediately followed or preceded by justification and more importantly,
critique of said justification. This is not to praise people who claim
that freedom (or openness) is a matter of efficiency rather than
morality -- they haven't avoided making a moral claim. Moral claims
about freedom and efficiency as top values have been relentlessly
scrutinized by moral philosophers and social scientists. Still there is
much more to say. Free knowledge movements probably have much to
contribute to the discourse, but we have to stop being satisfied with
straw man arguments and propaganda, even while acknowledging that such
have a place. Paths forward include breaking down and scrutinizing
``free as in freedom'' from the perspectives of various conceptions of
freedom and other values and objectives such as efficiency, equality,
and security. Doing so will make you morally smarter, more interesting,
and make it more possible for people and movements with non-freedom top
goals or different conceptions of freedom to join in the struggle for
free knowledge.

Opportunity cost. Participating in free knowledge movements often
entails filling one's brain with ridiculous trivia (e.g., about
copyright), developing one's skills to workaround underdeveloped systems
and institutions (e.g., administering one's own server,
{[}self-{]}publishing with little or no support for financials,
distribution, marketing), and self-exclusion from dominant venues and
tools. Each of these has a huge cost. You could be learning something
non-ridiculous, developing capabilities and competitive advantage rather
than engaging in a brutal exercise of de-specialization. One step
forward is to admit that these are huge costs, take them on carefully,
and avoid criticizing those who fail to fail to take them on, at least
not without acknowledging that they are costs rather than, or at least
in addition to being moral imperatives. Once admitted, free knowledge
movement actors might prioritize reducing these costs.

Scale. Free knowledge movements are often thought of as ``bottom up'' --
see phrases such as ``many eyes make bugs shallow'' and ``democratized
innovation'', the idealization of DIY, decentralization, and
contribution by individuals and small non-profits; and suspicion of huge
government and companies -- at best dominant institutions can be
``hacked.'' Now DIY and bottom up innovation and small non-profits are
vital and cool (well maybe small non-profits are only vital) -- but
alone, they are dwarfish and stupid. Huge systems and organizations are
not only corrupt and unjust -- they have huge economies of scale, deep
and specialized knowledge, and win markets and wars. Small scale free
knowledge actors are foragers who feel comfortable among their kin and
kindred, fearful of the farmers and their kings and armies -- and are
about to (on the scale of human history) be driven to extinction. If
freedom is important, freedom movements abhorring large institutions is
the ultimate stupidity. The path forward is clear -- the handful of
already sizable free knowledge organizations such as Wikimedia, Mozilla,
and Red Hat must get much larger, entrepreneurs (``social'' or
``for-profit'') attempting to create more free knowledge ``unicorns''
(we can count consumer surplus and other social values in the
``billions'' evaluation) encouraged, and sights set on taking the
commanding heights (e.g., mandating free knowledge through procurement
and regulation) rather than voluntary marginalization and hacks. One
conception of a stupid person or movement is one that consistently fails
to meet its stated goals, or consistently is outperformed by its
competition, effectively taking two steps back for every one forward,
with bonus for failing to realize this is what is happening. In this
sense dwarfish free knowledge actors are stupid, and will remain so
until they crack the logic of collective action, mostly through huge
free knowledge institutions, though other improved coordination
mechanisms may help as well.

Diversity. Free knowledge movements aren't very diverse, which
contributes mightily to the costs of joining and scaling, and thus
intelligence, in addition to missing out on intelligence benefits of
diverse perspectives documented elsewhere. Much has been written about
lack of diversity in free knowledge movements, and there is currently
considerable effort by various actors to increase diversity, so let me
re-confirm my biases; that is, make additional suggestions. Moral
certainty is bad for diversity. It is repulsive on its face, but also
allows continuing failure to make free knowledge concerns pertinent to
more diverse groups. Huge opportunity costs make participation feasible
only for the relatively privileged, self-limiting diversity. Lack of
scale makes free knowledge movements insular and non-diverse. Like
hanging out with culturally similar committed free knowledge hacks?
Great, you're in the right social club. Want world liberation? The cost
in the short term might be shedding some certainty, insularity, and
fear, and thus feeling stupid. It'll make you, me, and free knowledge
movements much smarter in the longer term.

Toxin. One topic endemic to most free knowledge movements is worth
calling out as an especially potent brain toxin: licenses. Yes they're
necessary for the most part given bad default knowledge governance. But
they make us stupid, over and above knowledge of copyright, patent, and
other regimes entailed. Identities are wrapped up in particular license
preferences. Consequential claims of license effects are strenuously
argued with zero evidence. No worked-out model, no empirical evidence,
whether from economics lab or natural (possibly instigated) experiments.
Anyone looking at these debates from the outside (unfortunately almost
nobody does so we're spared the richly deserved embarrassment) ought to
laugh at the level of evidence freedom observed. Emphasis on licenses is
morally ruinous. Developers, authors, etc. are placed in a privileged
position: supposedly freedom is the right of all, but creator choice is
lionized. The consequences are terrible too: creator choice is a recipe
for dwarfism. Licenses are a distraction as well from public policy.
Acknowledging again that licenses are necessary, the step forward seems
obvious: re-conceptualize licenses away from vehicles of creator choice
towards prototypes for commons-favoring public policy. This exercise and
actualization will make free knowledge movement actors much smarter --
we'll have to engage with the non-dwarfish implications of free
knowledge and actually convince people with other top policy concerns
rather than hide from them.

One way to decrease the stupidity of free knowledge movements is more
cross-fertilization and knowledge and tool sharing across said
movements. Stupid-making knowledge acquisition about topics such as
copyright and licenses ought not need to be re-experienced in each free
knowledge movement silo. Intelligence-building comprehensive criticism
also ought be shared across silos. Breaking apart the silos would also
increase diversity -- each has a different mix of participants, even if
they are also almost all biased in some of the same ways. While good for
the whole, a warning to individuals: attempting to learn about and
cross-fertilize multiple free knowledge movements might come at an extra
high cost to your intelligence.
