\subsection{Reconciliation}\label{reconciliation}

\begin{quote}
\hyperlink{hellekin}{hellekin},
\hyperlink{natacha-roussel}{Natacha Roussel},
and \hyperlink{pauline-gadea}{Pauline Gadea}
\end{quote}

Like a teenager discovering the shortcomings of the father, 21st Century
humans want to break free from a paternalist system that cannot address
complexity. They start looking after each other and invent new
associative institutions for solidarity, and take the responsibility for
their own future without waiting for the next false promise to come
true. In the dying liberal system, the promise of personal growth and
individual freedom is considered the key to a successful life and-or
entrepreneurship. In this context, however, individual freedom is often
understood as the capacity to do anything you like without
responsibility. In the upcoming social re-organization, stability is
grounded on free, voluntary association, and a new concept of freedom is
necessary to keep the system from running out of control. We must
acknowledge that with freedom comes responsibility. If ``with power
comes great responsibility'', political power brings the most
responsibility, therefore it must respect individual freedom in the
first place.

The antagonistic contradiction between global and individual freedoms
brings on the notion of choice and responsibility to create the balance
and resolve it at another level of reality. Gaining power is not anymore
a question of taking it, but to accept responsibility at a global scale.
Not only to accumulate knowledge but to learn to be human, and learn to
live together. The pathway to a different socio-political organization
starts with the deconstruction of the fundamentals of our civilization:
individual freedom is most interesting in all aspects when it is
measured with regard to the social constraints, it then becomes
productive of worthy social and collective outcomes. Each individual can
then root her personal development both in a local and global community,
therefore reflecting personal action to nurture both the personal and
the collective. Interdependence enabling self-determination can activate
personal freedom as a responsible asset. A severe impeachment to
self-determination is paternalism, a principal regulator of our
infantilizing civilization. It can be retraced up and until liberalism
and must disappear from a different organizational model if we are to
achieve global individual and responsible freedom, responding to the
injunction to ``think global, act local''.

Free culture is all about addressing this contradiction as it emerges
from this polarized tension. It produces the actual means and technical
tools both inspired by those issues and created to resolve them. But
free culture was born in reaction to the impeachment of
self-determination and it struggles to blossom beyond resentment. As it
rejects the paternalism of established institutions, it is harder for
free culture organizations to benefit from the synergies of
interdependence that could enable it to become a tangible way out of
dying social structures. It rises the essential question of scale of
organisazion, on which contributions here above express diverging
opinions.

A recurrent pattern in free culture and free software is the lack of
means to achieve stated goals, that ends up limiting the scope of
action. Proponents of scaling up to big entities, as well as proponents
of small, resistant networks need to overcome their differences: both
approaches present opportunities and caveats, both are complementary.
Large entities have easier access to capital, and can unfold economies
of scale, as long as their action is focused and directed. But that
comes at the price of slowness and a lack of resilience. On the
contrary, distributed networks must offset the costs of their autonomy
and their speed in line with a lack of funding that can be paralyzing.

Large entities are more likely to obtain public grants, as they can
invest in the time and skills required to write acceptable applications.
It involves technical and administrative knowledge and know-how that is
often lacking in existing solidarity networks. But such grants generally
allocate funds to tightly focused projects, serving specialized tasks
and positions. Meanwhile small networks are often divergent,
exploratory, involving multiple skills from a variety of disciplines:
this work cannot be covered by grants which impose accountable
production plans.

The question of how to enable complementarity between larger
institutions and more informal networks is one of balance between power
and agency. Public and corporate institutions naturally exercise power,
given their scale and position within the interdependent networks of
global society. But existing solidarity structures and systems enable
concrete actions within the communities themselves, often out of reach
of formal institutions. Not only the free culture movements need to help
and enable each other, institutional powers also need to accept letting
go of their trouble children, and enabling decentralized informal
networks to intensify their social ties beyond specialization and a
predetermined reading grid. Only then can we end infantilization and
become adults as a species: by cooperating responsibly as members of a
global society that embraces life, in all its complexity, uncertainty,
and affectivity.
