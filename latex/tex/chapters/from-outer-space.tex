\subsection{From Outer Space}\label{from-outer-space}

\begin{quote}
\hyperlink{the-big-conversation-space}{The Big
Conversation Space}
\end{quote}

\emph{An Imaginary Conversation between the Author and Alex Kurtzman
Regarding the Need to Base a Character in the new Star Trek Television
Series (premiering on CBS Television in January 2017) on Bassel
Khartabil}

\textbf{TBCS}: Alex, thank you so much for meeting today. I know you
don't have a lot of time.

\textbf{AK}: I don't know how you got into my office, but I´m intrigued
enough to give you about 5 minutes.

\textbf{TBCS}: Great. Well, first of all, congratulations on being
appointed executive producer of the new Star Trek series. That's a
tremendous honor, and it also carries a great responsibility, a
responsibility to use the show to call attention to contemporary ideas,
issues, events and people that can help pave the way for a better
future, a future where humanity has forged peace on Earth and can
explore the universe in the quest for new knowledge and culture.

\textbf{AK}: Yeah, those are some of the building blocks of the Star
Trek universe, sure. But we don't have the freedom to preach vague
ideologies. Everything needs to be packaged in a way that will attract
the most possible viewers. Do you have an idea that will help do that?

\textbf{TBCS}: I do have an idea, but before I get into that, I just
want to remind you that no matter what impact the studios have on the
decisions you make for the direction of the show, you are afforded more
freedom than you may realize. You may still be somewhat enslaved to the
pursuit of profit, but you can raise questions about power, you can
criticize authoritarian ideologies, you can present a future that
inspires people to work together now to m\textbf{ak}e some semblance of
it possible. You can do all of this without putting your life at risk.
The studio executives might not be open to some subversive ideas, but
you do not need to fear for your life, or that you could be arrested any
minute for even hinting at free and open discourse.

That said, what I came here to do is tell you that there is this guy
Bassel Khartabil who really needs your help, and I think you should base
a character in the new Star Trek series on him. He is a wizard computer
engineer, a compassionate and charismatic guy, and he has been
imprisoned in Syria because of his work advocating for an open Internet.
This makes for an inspiring character whose story reminds us all of the
freedoms we may take for granted in our everyday life, our everyday
future.

\textbf{AK}: Well, I do hope that we will present a future that inspires
some of the audience to do something meaningful, and I am curious to
hear more about this Bassel guy. But bear in mind that we have a lot of
the foundational characters already set.

\textbf{TBCS}: You have the whole senior staff figured out?

\textbf{AK}: Most of them.

\textbf{TBCS}: Do you have an engineer? Because the Bassel character
would have to be the engineer, the chief engineer. He has all the key
traits of engineers throughout the series: he is a brilliant problem
solver, passionate about technology, compassionate about people he works
with, dedicated to making the world a better place.

\textbf{AK}: The chief engineers in Star Trek have been dedicated to
their ships, not to making the world a better place. I mean, I'm sure
they're as interested as any other graduate of Starfleet Academy who
gets placed on a galaxy class starship, but that's not their focus.
Their passion is the ship.

\textbf{TBCS}: Sure, but do we really know that? I mean, in the context
of the show, the ship is their world, and they are dedicated to making
that a better place, or at least a place that is not breaking down.

And besides, his dedication to helping others is precisely the quality
that would make him such a valuable member of the crew. Because it's not
just that he is the kind of person who could maintain the coexistent
operation of a space station that is powered by the infrastructure of
three different species, like Chief O´Brien in Deep Space 9, but he is
also eager to share this knowledge with others and empower them through
it.

\textbf{AK}: Hm. We have been talking about how the technology is a key
gateway for audience interest, and having an engineer character who
helps facilitate that knowledge and understanding is an idea worth
tossing around. But, so what, these are commendable traits, sure, but
what is so unique about this guy in particular that would m\textbf{ak}e
him a compelling character that would keep audiences riveted and
interested?

\textbf{TBCS}: His backstory. He is a Palestinian-Syrian programmer, the
son of a famous poet and a gifted engineer, who started sharing his code
online for free and becoming involved in major internet projects like
Mozilla and Wikipedia, He started a hackerspace in Damascus, and started
Creative Commons in Syria. He vastly extended Internet access in Syria,
a country with a notorious record for Internet censorship and
prohibitively expensive Internet access. And his dedication to open
knowledge and sharing culture made him a threat to the authoritarian
government, so he was arrested.

\textbf{AK}: Is he still in prison?

\textbf{TBCS}: His whereabouts are currently unknown, he was moved from
his prison cell to an unknown location about a month ago, on 2 October
2015.

\textbf{AK}: I'm very sorry to hear that.

Well, I can tell you at least that we are interested in making some
reference to the refugee crisis, at least the concept of refugees. And
Internet stuff, like surveillance and censorship, are certainly hot
issues today and we intend to integrate them into some storylines. But
it is unlikely we will make any specific reference to Syria. This is
about outer space.

\textbf{TBCS}: But directly referencing what is happening in Syria via
this Bassel character is extremely important. Star Trek has always
engaged with themes that connect to current events (relative to the time
in which the series is made), and the war in Syria and its global
implications is easily the most significant event occurring right now,
and it's one that you have the power to impact.

\textbf{AK}: Again, this is a television show. Its intention is to
entertain people, not to stop wars. I have about 1 minute left and am
open to hearing more specifics. I am intrigued by this guy, for sure,
but I would need more of a hook in order to actually consider this.

\textbf{TBCS}: All right. I assume you have heard of Palmyra, the
ancient city in Syria that served as a vital crossroads of trade and
culture for millennia until many of its archaeological wonders were
senselessly destroyed by ISIS.

Well before Bassel was arrested, he was working on documenting the site
via photography and 3D models, creating a virtual reconstruction that
would allow people to learn more about the site and its history in an
innovative, immersive fashion. He could not have known at the time that
much of the actual site of Palmyra would be destroyed, indeed, at the
time this prospect likely seemed impossible. But today many of the
renderings he made for this project are the best surviving sources of
data about the site.

\textbf{AK}: That's incredible. Are these renderings or this data
publicly available?

\textbf{TBCS}: Yes, and they are in the public domain. There is a
movement, a community and a web site called New Palmyra where artists,
scientists, and designers are coming together to share, explore, and
build upon Bassel´s data and renderings, to virtually reconstruct
Palmyra's heritage and in so doing build cultural understanding that
transcends geographic and political borders.

\textbf{AK}: That sounds pretty well in line with Star Trek´s mission,
and like something that could make for a great holodeck program. And
since the files are in the public domain already, we would have
significantly more freedom to experiment than we would if we had to
construct them from scratch.

Well, I think it's been more than 5 minutes. I have enjoyed this
conversation and I will see what I can do. At the very least, I think we
can name a shuttle or an exoplanet after New Palmyra.

\textbf{TBCS}: So long as there is a Bassel riding in that shuttle.
