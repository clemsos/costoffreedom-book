\subsection{\#NEWPALMYRA and the Free Bassel
Campaign}\label{newpalmyra-and-the-free-bassel-campaign}

\begin{quote}
\href{../appendix/attributions.html\#jon-phillips}{Jon Phillips}
\end{quote}

\emph{Edited by
\href{../appendix/attributions.html\#patrick-w-deegan}{Patrick Deegan}.}

Bassel has been imprisoned for nearly four years, I believe it is about
1400 days now, but I have lost count; and since then we have been
running the \href{http://freebassel.org/}{Free Bassel campaign}. The
most depressing thing is that he has been missing for over a month now.
He was taken from the prison, and his name was removed from the list of
prisoners. We really really don't know where he is. He may even have
been kidnapped though it's more likely that the Assad regime has him in
a military prison. So that led to an acceleration in our efforts.

One project Bassel had started before he became a political prisoner was
the \href{http://newpalmyra.org/}{\#NEWPALMYRA project}. There are
actually several projects and ideas he created that have not yet been
announced, so this is the first of the many different projects we are
now undertaking to help call attention to his plight, as well as the
importance of his work.

The idea behind \#NEWPALMYRA was to recreate the ancient city of Palmyra
in 3D virtual reality. The \#NEWPALMYRA project is a new online
community platform and data repository dedicated to the capture,
preservation, sharing, and creative reuse of data about the ancient city
of Palmyra. The main idea is to focus on model quality first, and each
subsequently completed section will be released into the public domain.
We will release a master plan of the city and then a 3D model of the
city --- we want to keep moving forward on NEWPALMYRA, the city of
heroes that cannot be conquered. We will release all the data under the
Creative Commons Zero license, so anyone can do anything with it. We
already have contributions from different places in the world. Our hope
is to partner with other organizations like Creative Commons, MIT
Medialab, and the Barjeel Foundation in Dubai, who we hope will become
data providers and production partners on this artistic and scientific
project.

That's really the historical significance of the name PALMYRA, and we
are trying to embody that essence. We haven't announced the full list of
projects yet, but we'll begin by announcing artists and shows from
around the world about PALMYRA. So no matter what type of symbolic
destruction or act happens, and we hear about the terrible things being
done, we will do longer, better things. In fact, it's even more
transcendant: We BUILD culture. (They destroy culture.) We extend
memory. (Others forget.) We REMEMBER. We never forget about our friend.
But we're also not single-mindedly political in our efforts to build up
the city again. We hope it is built in as many different forms as there
are builders' hands. And to that end: we need your help as well.
Palmyra.org is where to join forces with us. And if you have any
particular skills or photos, please share them with us. We can use those
photos to create 3D models through photogrammetry.

There are two other projects that have been initiated, all linked to
NEWPALMYRA, that I want to discuss here. One happened in Paris, the
second took place near Aix-en-Provence. The idea is to write a book with
several creative cultural producers and software developers titled
``Cost of Freedom.'' That's something we talked a lot about with Bassel.
We have it done today. So he really initiated this idea as well, and
while the book as it is now has been written with a somewhat different
and more urgent focus, the core of the project remains consistent with
our original vision. The tech from projects---the collaborative,
multiscale, interdisciplinary, and international aspects of it, as well
as the actual method of production and technological content---will also
go into the book. And that's a powerful thing.

Part of this current idea for Cost of Freedom also comes from doing an
earlier event called ProtoCultural, which was first organized in Paris
in 2015. The idea was to get people together for two days and use the
time, community, and derivative data to then create and generate
artwork. Among the immediate fruits of that labor was an Artshow. In
another case, the artist Amad Ali created an optical installation from
the columns of the Temple of Bel. In deference to that but in a more
playful mood, Christopher Adams from Fabricators/Free Souls made a
\#NEWPALMYRA drink at the event. The press was there, and we were

getting a lot of attention and a lot of coverage. The reason was simple:
because it's such an outrage to destroy our shared heritage. So now that
we've done ProtoCultural Paris we plan on doing ProtoCultural Dubai.
Then we're going to do ProtoCultural Beirut in a couple of weeks. And
then ProtoCultural London. There are also several other cities we have
yet to announce. But we've had a lot of success scaling events, and
we're going to scale this to at least a hundred different cities around
the world. If you live in a city or even a town or any interesting
location, then let's do it. Let's do a ProtoCultural event. You share
with us; we share together.

Our next step of many leads us to Dubai. Dubai is a really amazing city.
For some reason, I have never really been present in Dubai. But I know a
little about the city, and it's interesting to see Gulf futurism---an
expression I borrow from our friend Sophia Al Maria. This is an apt
expression because maybe \#NEWPALMYRA will be like this, maybe we can
build it up, right out of the desert. We can raise it up. We can build
it in space. Or we can build it just online. Everyone's welcome in
PALMYRA. There's no people without land. There's no problem there. We'll
just create more land if we need it. So I think Dubai is an inspiration
for us because if you can lift the buildings like bar graphs to the sky,
then things can happen.
