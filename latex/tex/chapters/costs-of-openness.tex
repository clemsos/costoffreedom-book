\subsection{Costs of Openness}\label{costs-of-openness}

\begin{quote}
\hyperlink{tim-boykett}{Tim Boykett}
\end{quote}

This is a collection of notes about some thoughts on openness as a way
of working, living and acting. The summary might be that openness is
about conversations, about being able to discuss things, about not
sticking to your guns about taste, correctness, relevance and all that,
but about building communities of sharing, caring and being able to
correct one another's mistakes. Maybe openness is a state, not a
statement; a process, not a proclamation.

We do not do copyright every well at Time's Up: ``Copyright is
problematic. Contact for clarification'' or something similar is at the
bottom of many of our web pages. We do not think we can make a single
licence statement that will work. We would like to talk to people and
organisations about what they would like to do with the images, the
texts, the audio files. We were surprised when a huge image from our
work was used to announce the application for Linz to become the
European Capital of Culture, without asking us. It is nice to be so
appreciated that we are a beacon of Linz culture, but we ask you to talk
to us. A licence is possibly a way to avoid talking to one another,
openness is perhaps about encouraging us to talk, to think, to share and
communicate, not just announce.

Open academic publishing allows too much nonsense and badly written yet
often useful stuff to get out. Peer review does not stop this, but
quality reviewing does. This is a discussion between the author(s) and
someone who cares. A reviewer is a peer who should care. If one is asked
to be a reviewer, it is bound up with some work and some responsibility.
It is not a job of letting your friends in and keeping your foes out. It
is a job and responsibility, one of the rights and responsibilities that
comes along with the context of being part of the academic or research
community. It is possible to say ``either this is badly written, or I do
not care enough about it to develop an opinion'' as a way to pass on the
chalice. If no reviewer can be found who cares about the work, then
perhaps no one cares about it at all and perhaps it is not worth
publishing. Peer review means that the question of ``who are your
peers?'' needs to be answered. Who are they really? Who cares? This is
not about advertising or selling your work, about making people care,
but about finding out who does. I do not have the right to demand that
you care about my work. You cannot keep up with all the things that you
might be interested in, so unless you can trust that I have made
something that you care about, why would you bother looking at it?

Less is more. Fewer things to sort through. We have too many books,
publications, articles, white papers, etc, etc: how to find what we
need, let people know about what we do. We in the sense of all the
communities I am involved in, from Time's Up through the universities,
the research communities, the cultural communities and the world in
general. Not all forms of openness can help that, many will harm it.

Patents only help if you want to ``exploit'' the invention. If you just
care about doing interesting things, then being first is enough. Or even
just doing it. Patents have that secondary effect, that once the idea is
patented, we can all see how it works. So Patents are opening and
closing: I know how it works, but I cannot copy it commercially. Like
open source software: I was surprised to learn that commercial
programmers are not allowed to look at how something is coded in open
source software, in case they accidentally copy the programming
technique. So for them, making it open closes it. There is, of course,
the danger of reinventing the wheel (we have done that), wasted effort,
dead-end developments. That's fine. In the long run, we are all dead and
all the effort was futile. But in the meantime, let's keep it
interesting. Let's share ideas and experiences and find communities to
be involved in.

Open acknowledges mistakes and wrong directions. But we don't need to
proclaim them: the reason there is little interest in the ``Journal of
Negative Results'' is that failure often just means ``I cannot see how
to do this'' rather than the implied, or even believed ``this cannot
happen.'' Mathematics is a great place to investigate this. A naive
mathematician will say that something is obvious because they cannot
imagine why it cannot be true (and will use that as an argument!). This
might often be true, but it is not an argument. This is an enunciation
of ``common sense'' or ``intuition'' and mathematics is a machine for
breaking intuition. By doing the details, you might find out why the
statement is false. Or why it is true, not just because there is no
option, but because of something more interesting and useful.
Mathematics is about this openness in all its horrible, gory, intricate
detail. A mathematical paper is filled with long proofs because these
are the things that interrupt or confirm beliefs, hopes, steps to
results that are interesting. Openness here means that I open up my mind
and show you not only that I can do this thing, but how I do it, so that
you know that each time I do it, the answer is true. And thus you can do
it too. It is open and open.

In order to be relevant, mathematics needs two things: to be true and to
be interesting. One of the downsides of open publishing is that spotting
the interesting becomes harder, because there are no gatekeepers who
polish, edit, review and perhaps reject the ugly dross. We have to use
coding as a gatekeeper. Spotting references to Einstein, especially how
he is wrong, lets us know that a physics paper is probably
pseudoscience. The formatting of LaTeX as an indicator of seriousness,
Microsoft Word as a sign of an enthusiastic but probably misguided
amateur. But these codes are false, and occasionally as false as James
Lovelock's issues with scientific publishing from outside an
institution: because his address was not a university or company,
journals rejected his papers. Discussions were had and his papers were
accepted, but it was more effort, there was a gatekeeper that was using
inappropriate codes.

Openness has so many other branches. Money earnt, work done,
distractions allowed. In collective work, we often agree upon a ``basic
wage'' and share the work equally, something like from each according to
their abilities, to each according to their needs. But how many
innovators are independently wealthy and don't really need any financial
help? How many have artists have a side job as advertisers or share
brokers? Drunken writers write about drinking, not about what they do to
actually pay the bar tab and postage for their manuscripts. Academics
have tenure to allow them to undertake long projects. Or stay at home.
Or start a business. Or hide, tutor school kids, write a science
communication novel or a million other things. Are these distractions,
or are they desired tangential outcomes? Do we need transparency here to
know what is going on? Or does that break trust? How much box-ticking
and metric analysis is needed to ensure that ``public monies'' are being
correctly spent on science, humanities, culture and the arts? Are the
numerical results of bums on seats and webpage views actually useful, or
is that just another coded gatekeeper? If you can get through the dross
of the application, then you are serious enough to be able to make it
happen.

Perhaps transparency breaks trust. Perhaps openness creates not just
abundance but waste. In the sense of ``There's no such thing as waste,
just stuff in the wrong place.'' It is probably worth keeping a lot of
things out of the public eye, of not sharing every little detail on a
blog or a series of explanations of your theory of everything, or your
theory of everything else. Who are your peers, who are your colleagues?
If you have a question or a new idea, formulate it properly. You might
find the answer yourself while formulating it (Oh, that's what I
meant!), you might realise that the idea breaks once it is communicated
or becomes trivial (Ah, there are none of those to worry about). Then
talk to your colleagues, your community, the people who know you and can
help get over the first hurdles. Only then is it worth taking your idea
to a larger group, your peers. StackExchange and other places are filled
with comments that a given question is a duplicate of a given question,
that the questioner is wasting time and space by not doing their
research. If I want you to invest time in reading my question, you need
to trust me that I have bothered answering the question already. That I
have looked in all the normal places, tried the standard solutions. If I
want to revolutionise gender theory, then I need to have read enough
background, not just thought about it a bit and been excited by an idea.

Paul Erdos is an acclaimed mathematician, who would arrive with the
statement ``my brain is open'' and work with colleagues on problems
before travelling onwards to the next stop on his never ending journey.
This openness led to him being the most published mathematician in
history. His case is rare. The web is filled with examples of extremely
smart, well-meaning people sharing their complex and intricate
examinations of ways to improve the world, from engineering systems
science analyses of climate issues to disaster relief planning. However,
the absolute openness of their sharing means that every idea that
crosses their well-fed minds gets deposited in the collection, pages of
PDFs, hundreds of blog posts, hours of video lectures: too much! It is
said that mathematicians are cheap. They require paper, pens and a large
wastepaper basket. This process of disposal, of winnowing out the dross
and keeping the good stuff, is the core of good work. If only I would
learn that myself.
