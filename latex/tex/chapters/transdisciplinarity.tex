\subsection{Transdisciplinarity}\label{transdisciplinarity}

\begin{quote}
\hyperlink{melanie-dulong-de-rosnay}{Mélanie
Dulong de Rosnay}
\end{quote}

I fell under the spell of sharism in 2003 when I started the Creative
Commons France chapter with the full support of my then Ph.D.~advisor
and the director of our research center. Since then, my participation in
the movement has landed me a blissful life with lifelong friends, love,
and several paid jobs and grants both in my country and abroad with
lovely, smart, dedicated and gifted people animated by the values of
open access, open science, open licensing, peer production, the public
domain and the commons.

In 2007, several teams coordinated by Creative Commons Italy received a
grant from the European Commission to start Communia network in the
public domain and support our work. All this provided opportunities to
have a political impact and travel. It is possible to develop serious
research and policy contributions with a network of amazing colleagues
all over the world, people coming from diverse backgrounds who share
similar ideals.

The cruel detention of Bassel Khartabil reminds us of the incredible
luck of living in such a privileged environment with freedom of
expression. My only social cost has been exclusion by conservative
people from whom I needed neither approval nor friendship, and this
doesn't even happen so much anymore since openness is becoming more
politically correct and even hyped in Western culture.

To newcomers wondering if the cost in terms of time and efforts is worth
the involvement: it is nothing compared to the inspiration gained and
the joy and pride of contributing to a global movement that is
developing positive alternatives to enclosures, and promoting social
justice, freedom and access to knowledge, information, culture and
education, good food and medicine.

Even though some of us are techno-idealist, our work is not, neither is
it economically insane, but rather highly political and ideological.
Freedom of knowledge and circulation are battles to win over the
corruption and censorship of those whose addiction to unlimited
commodification, unsustainable growth and a vision of development based
on globalized extractivism that prevents personal and collective
development and the right to a good life for 99\% of the population.
