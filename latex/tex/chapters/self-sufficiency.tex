\subsection{Self-Sufficiency}\label{self-sufficiency}

\begin{quote}
\hyperlink{pauline-gadea}{Pauline Gadea}
\end{quote}

I decided 2 years ago to leave jobs in the media to go learning how to
make cheese. I saw this as a step towards essential freedom.

Beyond choosing a life much more in touch with nature and craft, it
would give me the freedom to carry around with me the ability to
eventually fit with a concrete valuable knowledge into communities that
aim at producing their own means of subsistence as much as possible,
maybe creating my own means of subsistence in the end. It represented a
step which would make it possible for me to build a life outside both
the mainstream work system and the food system. I see both of those
systems as freedom down-takers. My statement was more or less ``I want
to create something I would be proud of with my bare hands in a settled
place, which would help to tend to freedom for me and a small community
I choose. If not, I feel like I m neutral in the best case scenario in a
path to global freedom, feeding a mass system which deprives it in the
worst way of looking at it.''

This vision, of freedom linked with self-sufficiency in food supplies
and self-determination in terms of human organisation, is commonly
shared, at least as an ideal goal, but is also seen as quite extreme in
terms of the changes it requires for most people's ways of living.
Making a step towards it was a way to challenge my own motivation, my
own limits in relation to this fantasized ideal of real, deep, freedom
and its connection with a rural life.

In order to learn properly, I had to deal amongst other things with the
traditional farming work culture as an employee, a reality which was
further from the concept of freedom than I had experienced in all other
work situations, in terms of hierarchy, of hour-based deadline pressure,
of physical commitment. I didn't fit in but I still had to learn, and
earning money in the process also was quite essential.

You see your friends obtaining more freedom and self-realisations by
more classical means. Mainly by just mastering their work field little
by little, you see them having little by little better salaries, wider
responsibilities, recognition and range of action in what they are
doing. Then you start to wonder why you have to make it so
complicated\ldots{} I questioned my choices. Of course, making cheese
does not provide the same kind of freedom I was after when I decided on
this change of life, but it still gives the comfort and confidence I
might need for any future achievement.

In the end, setting a precise, high goal of freedom as a core
preoccupation in my life as a starting point hasn't led me to more
actual freedom (yet), but it obliges me to ask myself very often what is
that I'm doing and why am I doing it. Answering those questions makes
the commitment deeper and slowly creates the connections I need to live
a life closer to my ideals, like a vow I made that forces me to be brave
when I feel insecure about what I am doing, and making it silly to worry
about where I am going to keep those three pieces of furniture for a
while.
