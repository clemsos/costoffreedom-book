\subsection{\texorpdfstring{``Freedom To'' vs. ``Freedom
From''}{Freedom To vs. Freedom From}}\label{freedom-to-vs.-freedom-from}

\begin{quote}
\hyperlink{martin-paul-eve}{Martin Paul Eve}
\end{quote}

Unlike Bassel Khartabil, the cost to me, personally, of my
free-knowledge work has been cheap. I have not paid with my freedom. In
fact, I have been incredibly privileged to have conducted my work in
creating free and open systems for the dissemination of scholarly
knowledge in a geographical space (the UK and the British university)
and political time that for the most part actually rewards such
undertakings. If I say there is a cost, I feel it is a difference almost
of type by comparison, rather than of degree, with respect to the price
that Bassel has already paid.

But there is still some way to go, even in my privileged world. For the
most part, academics are assessed on their publication record in a
recognised disciplinary space, publishing with known proprietary
publishers. There are very few positions available for the practical
implementation of change in the academy (praxis). This is so to the
extent that Kathleen Fitzpatrick, a fully tenured professor in the
States, quit her post to work on publishing initiatives at the Modern
Languages Association. Fitzpatrick wrote: ``This is of course not to say
that one can't change the world from inside the protections of tenure.
But I do think that those protections often encourage a certain kind of
caution, certainly in the process of obtaining them, and frequently
continuing long after, that works against the kinds of calculated risk
that a chance like this requires.''

Even in my own academic publishing, though, there is a double bind. Many
of my colleagues continue to find (or at least believe) themselves torn
between publishing openly and having a career in the university.
Dissemination and assessment find themselves in conflict because
proprietary publishers own most of the venues for academic
dissemination. And hiring panels look for the books published by the
brands whose quality-control procedures they trust. But if those
procedures and trusted systems are owned by entities whose business
models depend on selling commissioned copies, then despite the fact that
academics can give away their work (because they have a salary) this
knowledge will remain imprisoned.

Even worse, this coercion (as I see it) to publish in known brand and
usually-proprietary venues as a proxy for hiring in the university is
defended as academic freedom (the freedom to choose to publish where one
wants, rather than being told to publish openly). Certainly, it's done
by ``soft power'' and a reputational/symbolic economy, but I did not
feel free when I had to publish my first book with a commercial press.
I'm still grateful to them because I needed the book for my job. They
did good work on it and I can't fault the people who helped me there.
But few people can actually read that book now because it is so
expensive. I signed away the copyright as the price for a job. In an
ideal world, I would have published this openly.

So, even as individuals (such as Bassel) fight for their true personal
freedom that was taken away because they developed open-source software
and facilitated freedom of expression, people around me continue to
claim that it should be their right to lock knowledge away and that this
is a freedom for them (see Cary Nelson's article in Inside Higher Ed.
for an example). I do not think it should be. Academic freedom in its
real and proper definitional sense is important (the right to speak
truth to power) but we should not demean it by saying that it is about
one's right to lock knowledge away from those who cannot pay.

When I say things like this, I am told I am anarchistic, that I want to
destroy tradition, and that I am somehow an enemy of quality in
academia. I have also been told that this coercive soft-power structure
of proprietary publishing doesn't even exist (usually by people who
haven't tried to get an academic job in the last decade). It does exist,
and I am not trying to destroy academic publishing. I am trying to make
academia and academic publishing the altruistic spaces of
knowledge-sharing that they should be. As academics, giving people
worldwide the freedom to read our work should always take precedence
over our personal benefit from publishing in closed venues. I have not
always been able to negotiate this cost successfully so far, but I will
not defend my self-interest as a ``freedom'' when there are people who
have really lost their freedom for this cause.
