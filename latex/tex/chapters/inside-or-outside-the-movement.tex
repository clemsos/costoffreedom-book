\subsection{Inside or Outside the
Movement}\label{inside-or-outside-the-movement}

\begin{quote}
\href{../appendix/attributions.html\#john-wilbanks}{John Wilbanks}
\end{quote}

Working in the free knowledge movement may mean working in a space that
is better fitted to contemporary technology, but it also means working
against several dominant themes in contemporary society and regulation.
Most of our societies prize fences, whether through copyright or patent
or contract or just simple withholding of secrets. Investors prefer
fences, and universities reward them too. As a result, working in free
knowledge is often a fundamentally transgressive act, politically and
economically. And transgression against dominant social concepts comes
with so many different costs.

There's a cost to explain free knowledge, because it has to start with
what's wrong with closed knowledge. That comes at a cost of having
friends or family understand the job, with questions like ``why do you
keep working on this when you could make so much more money somewhere
else?'' There's a cost to always being the outlier in a ``normal'' room
of professionals, working against the gravity that defines normal for
everyone else. There's a cost in constantly looking for funding when the
dominant capital systems don't reward or pay for freedom. There's a cost
in always feeling weird, always feeling like the power systems want you
to lose.

It's not unlike being in a startup religion, except there's actually
evidence for the benefits of free knowledge.

There's also a cost within the movement, one we don't talk about much.
When we do actually all get together, and for once we're not
transgressing against the ``rest of the people in the room,'' we have a
nasty habit of judging each other, fighting each other over details that
the rest of the world doesn't even recognize. I've been guilty of this
in the past. It's just so wonderful to be able to debate our work with
others who agree with us that it's easy to get into the details, and all
the passion we bring to changing the dominant social system suddenly is
focused on those who we agree with the most.

This isn't an unusual cost. In fact, it's one of the most common costs
of any social change movement. But it's the highest one, for me. The
only advice I have is: we're in this together, those of us who care
enough, those of us who see enough. It's easy to take that passion and
turn it against ourselves, but that's a target that only helps the
closed knowledge system maintain itself.

I've worked on recognizing that all of us, from the most strident
backers of the public domain to those who embrace non-commercial
licenses, from a total open commons to a network of managed commons,
have way too much in common to subscribe to a purge mentality within
free knowledge. I'm a lot less strict about applying definitions of
freedom to people -- those definitions are for knowledge objects! And
I'm a lot more inclusive of different opinions within the free knowledge
movement than I used to be. It means that at least I'm no longer paying
the cost within the movement, and I'm reserving all the resources for
the costs outside the movement.
