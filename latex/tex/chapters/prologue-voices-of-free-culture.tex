\subsection{Voices of Free Culture}\label{voices-of-free-culture}

\begin{quote}
\hyperlink{clement-renaud}{Clément Renaud}
\end{quote}

\emph{This book was written in Pourrières, France, in five days, from
2nd to 6th November 2015.}

Just two weeks before, I got a phone call from a friend, asking me to
help bring attention to the plight of Bassel Khartabil, by organizing
this book. We jumped into the project instantly, starting to pull people
together, authoring web pages and open calls, sending emails and calling
everyone we could think of.

The book you are reading is the result of this emergent process, based
on friendship, internal networks, and external publications.

It originates with our friend Bassel, suffering in a Syrian jail that
has taken him away from us. I have never met him, but I am calling him a
friend because I know from all who have known him that I will have a
good time meeting, talking, and working with him.

In the small group of ``free culture'' we tend to regard each other as
friends. We all feel committed to a common mission. For this book, we
made an open call to those who have ``been fighting in the trenches'' of
free culture. That sounds like an overstatement for most of us who are
not in jail but are instead mostly writing, coding and taking part in
interesting projects, enjoying our freedom.

Thus, when we call for a reflection on ``the Cost of Freedom'', we
suddenly appeal not to our group and our mission, but to each individual
that has been part of it. Instead of preaching the values of a whole
system supposedly based on commons and sharing, we target people in
their daily lives -- those who have suffered from loneliness,
questioning, bankruptcy, burnout, exploitation, and even from seeing
friends and partners suddenly missing, just for having been a part of
free culture.

This book is not a statement about freedom and culture; it is a primal
scream, the sum of our questions and desires. It is the raw expression
of our lives. It talks about what is ultimately made through the dream
of free culture: us.

This book is dedicated to Bassel Khartabil and to all those that will
recognize themselves in the stories told in these pages.
