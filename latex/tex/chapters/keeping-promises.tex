\subsection{Keeping Promises}\label{keeping-promises}

\begin{quote}
\hyperlink{lawrence-lessig}{Lawrence Lessig}
\end{quote}

Presentation by Lawrence Lessig, CC Global Summit 2015, 15 Oct 2015,
Seoul, Korea Edited by Christopher Adams, 3 Nov 2015, Pourrières,
France.

It's important for us elders to remind you kids of where you come from.
The Creative Commons project was the failure of a legal action. When I
was at the Harvard Law School in the late 1990s, Congress passed the
Sony Bono Copyright Term Extension Act, which extended the term of
existing copyrights by 20 years. We brought a lawsuit on behalf of a man
named Eric Eldred, an online publisher who wanted to publish the poems
of Robert Frost, which were to pass into the public domain, and would
have passed into the public domain, had Congress not extended for the
eleventh time in 40 years the existing terms of copyright.

As a law professor, as someone who had no desire to be an activist, I
learned of Eric Eldred and reached out to him to say, ``Why don't we
challenge this decision by Congress, because it seems so plainly
inconsistent with the idea of copyright for a limited time?''

We brought his case all the way to the Supreme Court, but just before we
got there, Eric Eldred said to me, ``Look, I appreciate what you're
doing, but I don't think we're going to win, and I don't want this just
to be a lawsuit, so I want you to promise me you will start a foundation
committed to the Commons.''

I was convinced we were going to win, so I thought, ``Okay, I can make
that promise, because if I win I don't have to start the foundation.'' I
made the promise, but then I lost the case in the Supreme Court. That
defeat gave birth to you, because once we lost, I had to deliver on the
promise that I'd made to Eric Eldred, and so a number of us sat down in
some offices in Harvard, and figured out how we would build what would
become the Creative Commons.

The proudest moment I remember from those early days was the way we
brought a young technical community into what seemed to be just a legal
argument. One of the early victories for me was persuading a young boy
of 14 or 15 years of age, named Aaron Swartz, to become the technical
architect of the Creative Commons, in 2002. It took a little persuading,
but I told him that this is what he had to do.

Later the relationship of me telling Aaron what he had to do reversed
itself. In 2007, I was finishing my last book on copyright and Internet
policy. Aaron came to visit and asked me what I was working on.

I was very proud to show him my book, and tell him about my first TED
talk. Then he asked, ``Why do you think you're going to make any
progress on copyright and Internet policy so long as we live with a
deeply corrupted government?'' I told him, ``It's not my field, it's not
what I do.'' He asked, ``You mean as an academic?'' I said, ``Yes, as an
academic. It's not my field. I am a scholar of copyright, and the
Internet.'' Aaron said, ``Ok, but what about as a citizen?''

What he did at that moment was to shame me into leaving this movement, a
movement that he had joined when I shamed him into building the
architecture of Creative Commons. He shamed me into leaving that
movement to take up a fight which has grown and has consumed my life,
and consumed my life right at the moment when many of us feel we failed
him, when he felt the burden of fights that he was in, in such a
profound way that he had to take his own life.

That transformation led me away. But there's nothing that gives me joy
like looking back at things that I had something to do with starting,
and seeing them flourish, and to see the spread of ideas which the
Creative Commons community has carried forward.
