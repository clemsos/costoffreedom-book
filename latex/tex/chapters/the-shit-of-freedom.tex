\subsection{The Shit of Freedom}\label{the-shit-of-freedom}

\begin{quote}
\hyperlink{giorgos-cheliotis}{Giorgos
Cheliotis}
\end{quote}

Somebody once told me that freedom is one of these words you can't
define without it becoming self-referential. A person will usually start
a sentence with ``freedom is when you're free to\ldots{}'' and their
minds will hesitate for a moment: a brief, unsettling glimpse at the
turtles that spiral all the way down. Few can stare into the abyss for
long, so they will quickly stumble back to the comfort of the known and
pick their favorite from a laundry list of personal wishes, desires, and
learned ideals, except perhaps if that person is an academic, trained in
words and systems of thought. There's probably many valiant attempts at
a definition out there, and I'm sure that on this very day a
Ph.D.~student somewhere is doing a literature review on freedom and is
insanely bored with it.

It is like that with universal, deeply rooted, almost primal human
desires. It is like that with freedom. We think we know it when we feel
it, we sometimes know it when we see it in others, but the words are
hard to come by. We struggle to produce a concise definition; we
struggle with the very concept of it, and at some point in our lives we
wonder how free we really are, and what freedoms are perhaps worth
fighting for. This, then, is the greater cost of freedom: its pursuit. I
guess what I'm saying, and I know this won't please the reader, is that
our natural state of being, for most if not all of human history, has
been NOT free. It starts with family; then our boss; the state; the
market; even our partners and friends.

Yet, for all that socialization entails, for all the grooming into
conformity and compliance, for all the rules that we impose on ourselves
and others, we will rebel many times over. And every time there will be
a price to pay. Some will pay the ultimate price. My friend Bassel, for
example, he pursued freedom of information in a land that couldn't face
itself, let alone a free spirit that soared above its arid lands. Like
Aaron Swartz, who pursued freedom of information in a market that trades
in commodities and not ideals; like Chelsea Manning, who pursued freedom
of information in a land that associates freedom with weapons, war, and
the power to abduct and incarcerate. Like the founders of Pirate Bay,
whose names I cannot recount, but whose services I have often used for
my teaching, research, and entertainment; and like countless others, who
will never make headlines. It is a sign of our times, that some of us
seek freedom in information, and that some will pay an inordinately high
price for it. Our age is the information age. And we are not as free as
we think.

I could leave it at that. I meant to write a brief commentary on the
cost of freedom. It's nothing special, but it's kind of neatly wrapped
up there. However, I felt the need to say something more, something more
personal and probably more important. To say it out loud, and ruffle
some feathers: first my own, as I'm getting out of my comfort zone here,
then those of some prancing peacocks in the free culture / FLOSS /
digital rights scene, and the technopreneurs peddling freedom for
dollars and fame. You see, dear reader, there are other, hidden costs to
the whole endeavour of digital liberation\ldots{} they're everywhere,
inside and outside, in movements and in people, such as disillusionment,
waste, the cost of stupidity, as a friend put it. I warn you now, this
will get ugly. It needs to be. If you know me, you'll know I'm rather
measured in my words and actions, even if not docile. But here I won't
be. You'll be offended. In fact, I hope you will be. As much as I also
seek the validation of others, I will now pay the cost of expressing
myself freely. With little inhibition.

And that is because, in many of my efforts to engage productively with
the project of digital liberation and assorted anxieties of our time,
there you were, my friend. The free software programmer that grossly
overstates their contribution; the free culture evangelist who's in it
for personal gain and will happily privilege their culture over
everybody else's; the opportunist entrepreneur and peddler of freedom
through code. You have spent a fair share of your life building things.
Building tools, networks and communities. Instigating projects and
influencing people, converting them to your cause, forging friendships
and partnerships; gathering resources to build the world you desire. You
are well educated, and you understand networks. In short, you have super
powers. You have gained the respect of many and probably made some
enemies along the way. You have a posse, your own personal echo chamber.
You come to events and gatherings filled with the contagious energy that
we all love you for. You speak of awesomeness, projects, personal
freedom, a friend in peril perhaps, a worthy cause for all to rally
around, a call from the White House, a famous dissident, a Nobel
Laureate, a Saudi prince. Networks of power, culture and code. You
gather resources. You make phone calls. You entertain and motivate. You
sound important. There is something about you. In your presence, names
drop like flies and jokes fly like bullets. You're awesome. And you're
so full of shit.

You're always on the go. You forget things, you complicate things, you
exhaust yourself and others. You rarely forget to pack your ego, though.
Sometimes you manage to squeeze it all into your suitcase, with a
toothbrush, t-shirts, and chargers for your beloved gadgets. But your
ego is so large and overbearing, you couldn't possibly fit it in your
luggage at all times. So you wear it on your sleeve. You armour yourself
with layer upon layer of ego steel. You prance about, crack a joke, or
two\ldots{} or three\ldots{} because you're awesome. You make a nice
gesture, make a plan, and seek the admiration of those around you more
than anything else. You make more promises than you'll ever be able to
keep. You make more plans than you will ever follow up on. You make
things fun. You make people believe in themselves. You make them believe
your shit. I think you believe it too. I'm happy to know you. But you're
so full of shit.

So is your posse, that echo chamber you've built for yourself which
reinforces the best and the worst in you. Together you peddle freedom to
make money, to peddle more freedom, to make more money. Sound familiar?
Yeah, it's what the US government does to the world.

What are the means and what is the end in what you do? I doubt you know
the answer. I only observe the briefest moments of reflection from you
and your buddies on what it is we're doing here. Faint rays of meaning
in a cloud of technobabble, freedombabble and babble babble.

You know I love you. You know I want to. I am charmed by your presence,
laugh at your jokes, and I have made some vaguely awesome plans with you
in the past. I too am a rather privileged white male who enjoys the
globetrotting lifestyle, the random jokes, the occasional debauchery,
the endless speculation over the next big thing, the code that binds us
all into one super-network of super-friends. But knowing you a little
too well, and being less gullible than I once was, I can see right
through your bullshit. And there's so much of it. Sometimes I know
you're trolling. Sometimes I wish you were. Sometimes I think you're
just trolling yourself.

I have made mistakes in the past. I have missed deadlines. I have failed
to meet goals. I have disappointed others. But I am trying to make peace
with that. I try to speak less and listen more, focus on what's
important, to be strong without being an ass, to be there for others, as
much as I also need others to be there for me. To respect others, give
them room to breathe, follow their lead when they know best, lead them
when they ask me to, work with them, not have them work for me under
false pretenses. Still, I fail often. Too often. I am aware of that. Are
you? And what do you do about it?

I'm not ashamed of my failings, not too much at least. Nor should you
be. I know you sometimes are. Is this why you never stop? Is this why
you're always on the move and never ever shut up? Is this why you hate
stillness? Self-reflection is a downer, right? So is contemplating the
cost of freedom and the vapidness of so many of your projects. I know
the drill: invest in a huge number of things, because only one or two in
a thousand will succeed and make you somebody you so desire to be. Of
course, this only works if you can keep the cost ridiculously low and
make sure you contribute next to nothing to any single project yourself.
It's all about long tails and downside risks and cheap labor. I've read
the blogs and talked to the people you talk to. I probably read the
books you didn't, because you were too busy working on your sales pitch,
or curating your posse.

So there you are again, giving some of yourself so you can take much
more in return. Exploit the resources, the goodwill, the gullibility and
pain of others. Voluntary, cheap labor, free software, free licenses,
free content and free beer. Sometimes you'll pay, sometimes I'll pay,
but we gain nothing other than a few laughs and some bruised,
hypersensitive, needy egos. I have to wonder if you ever built anything
of note yourself. I can't tell anymore what's honest about you and
what's dishonest. What's real and what's pretence. You'll find a cause
that will serve your needs, you'll grab it and run with it to investors,
conferences, seminars, workshops, roundtables, parties, art galleries,
hackerspaces, incubators\ldots{} anywhere you can sell the cause and
find believers in something you don't believe in yourself.

Because if you did, you'd be fucking serious about it. You'd give up
that bullshit venture capitalist mentality that's there to make money
for the few and feed your ego in the process. You'd see how it's all
exploitative and stupid, and then you'd be truly embarrassed. Very
embarrassed. You would understand that when you ask me to join an effort
being made in the name of a cause I strongly believe in, and then you
make a mockery of that effort, I feel stupid for even trying. And then
I'm angry. So angry that I may say nothing, out of respect for those
around us who are truly trying to make something of the moment, not only
for themselves but also for others -- like Bassel, who has sacrificed so
much for something he believed in, with nothing in it for himself.
Instead, I'll pour my anger and disappointment here. And my love.
Because without love, I wouldn't have cared to write this. Without hope
that you will read this and have a ``FUCK ME!'' moment, I wouldn't have
bothered. Because you're awesome. And you're so full of shit.
