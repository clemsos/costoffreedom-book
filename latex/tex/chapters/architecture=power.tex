\subsection{Architecture = Power}\label{architecture-power}

\begin{quote}
\hyperlink{stephanie-vidal}{Stéphanie Vidal}
\end{quote}

Je ne bâtis que pierres vives ce sont des hommes. I only build with
living stones, those are men. --Rabelais, French humanist (1494-1553)

The Code of Hammurabi, a basalt stone covered with cuneiform script,
preserved at Le Louvre in Paris, is recognized as an important artifact
for both art and history. Erected by the King of Babylon, Hammurabi,
``protector of the weak and oppressed'' circa 1792--1750 BC, the Code is
the most complete legal compendium of Antiquity, written even before the
Biblical laws. Emblematic of the Mesopotamian civilization, the stone
embodied the Law into a single, indivisible object.

On rocks, monuments, or in city topologies, societies through the ages
have inscribed their rules into architecture. Today, we no longer
engrave laws onto stone, but architecture remains powerful at a symbolic
level. ISIS, as a recent example, is destroying ancient temples in
Palmyra and elsewhere in the Cradle of Humanity, because they recognize
its representation of older culture. By desecrating these old monuments
and broadcasting their destruction online, ISIS wants to show the world
that it is destroying the memory of a period before the Prophet, and
deleting the cultural symbols of Bashar Al Assad's power, making way for
their new Caliphate.

Whether a smooth basalt stone, a Hindu temple, the Eiffel Tower, the
pentagon, the Twin Towers, or your own house, architecture is always the
manifestation of a system. A signifier of values, it contains a will to
express the inherent power it represents. Building or destroying
architecture is a mechanism for power to send a strong message to its
audience.

Digital tools now allow further options. They can express the
willingness to rebuild, and to oppose brutality with creativity; not
with real stone, but with people that are the ``living stones'' of
Rabelais.

The \#NEWPALMYRA project is at the cutting-edge of this international
movement. Born out of the emergency of the Syrian crisis, the
\#NEWPALMYRA project is an online community platform and data repository
dedicated to the capture, preservation, sharing, and creative reuse of
data about the ancient city of Palmyra.

In this project, the power engaged is the power of people to channel
their outrage and create hope through action. Aiming to virtually
reconstruct Palmyra's cultural heritage, gathering data and knowledge,
\#NEWPALMYRA is an expression of a collective consciousness.

People often make an opposition between the digital and the real but it
is a pointless statement: the digital should be considered as an
actualisation of real desire, as a space-and-time singularity where
everything and everyone (even the dead and the missing) can be a
presence for someone else.

We are now living in a world where the digital is omnipresent, and where
power is embodied in virtual and intangible architectures, and code
still comes from stone: computers are produced from geological sources
such as quartz or coltan.

This so-called virtual place is made out of real materials and is based
on infrastructures, such as data centers, embedded in our ecology. The
Internet and all complex information systems are real architectures, and
so are also an expression of power: their structure is not pre-existent,
but created intentionally by their designers.

We are all evolving a world made of digital and spatial layers, where
technologies are now able to follow and record our traces. The German
architect Jürgen Mayer H. has expressed this contemporary double effect
in his work, documenting where inhabitants leave traces of their
presence over the ground and walls as they pass. According to Mayer H.,
``there is no such thing as a naive or innocent surface.''

Archaeology is the science of identifying and studying ancient traces
now preserved in ground or wall, to understand what or who left them in
their present time.

In the network, we are living in the traces we leave in our everyday
lives, using social media, producing or sharing content, taking pictures
or being tagged by others, having a real-time narrative approach to our
lives, valuing our past and accomplishments, confessing to all our
followers or stalkers what we were, are, and want to be with words or
metrics, and in which kind of world we wish to live.

In ancient times, the worst punishment that could ever be pronounced
over someone, even worse than death, was called Damnatio Memoriae. This
post-mortem sentence given to a public persona implied that their name
would be erased from all public monuments, and their statues pulled down
or destroyed so they would be forgotten by the people over time. But,
today, we cannot be forgotten or discreet because of the constant traces
we leave on the Internet.

The Internet era is the age of the Chiaroscuro, where shades of
intention coexist: the impossibility of being forgotten and the craving
for attention, the use of the same tool by some to preserve the history
of ancient Palmyra, while others use it to delete the past and broadcast
their terror and destruction; the desire, through technology, for both
individual empowerment and mass surveillance.

If navigation into the digital spaces is no longer naive and can be used
for surveillance, what about a system where the law could judge your
intentions as well as your actions? Could we be tracked and trialed for
moving freely within it?

This awful and highly complex current war is devastating Syria, harming
its cultural heritage, and persecuting its ``living stones.'' What is
happening there shows the international community that people are being
tracked for expressing their will for freedom, be it with something as
simple as a ``like'' on a Facebook Page, or more arduous, such as
founding an entire hackerspace.

This cruel reality has to sensitize us to the power of information
technology, that it can be used, like any tool, for good or evil. The
ancient Greeks were aware of the dual nature of the pharmakon: in a
coercive system, the way you live or the path you take is enough to make
you suspect, and those systems punish intentions and actions equally
because of their potential for disruption.

The other lesson we, in the ``free world'' (where we don't have to be
afraid of being shot by a hidden sniper), have to learn, is that liking
a Facebook page, or founding a hackerspace, does not have the same cost
for people living under a different system than ours. For us it's just a
social interaction, for others it's a social action that can have
terrible effects on their lives or the lives of their loved ones.

We have to find a way to move freely in our minds and within the
systemic information architecture for it to remain a tool that can
empower the people and not enable a few to reduce freedom, enact
personal censorship, or jail those they perceive as threats for their
oppressive systems.

Ancient Greek orators used to create mental and imaginary architecture
as mnemonic techniques, to remember their long speeches so they could
easily express their arguments in the Agora. Today, technology helps us
acquire knowledge, express our opinions, and remind us that freedom is
not something slight to be taken for granted.

We are all at risk if someone more powerful than us doesn't want us to
move anymore, in the streets or on the network, so we all have to ask
ourselves: What is the price we have to pay to inhabit this new
architecture we are collectively building, and what do we have to do to
preserve our freedom within it?
