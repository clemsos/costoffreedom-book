\subsection{Free Culture in an Expensive
World}\label{free-culture-in-an-expensive-world}

\begin{quote}
\href{../appendix/attributions.html\#shauna-gordon-mckeon}{Shauna
Gordon-McKeon}
\end{quote}

\begin{quote}
``Free as in speech, not free as in beer.''
\end{quote}

How many times have you heard this explanation of free software? It's
cute, catchy, and a little too glib. After all, nothing's ever that
simple. But this phrase is more than an oversimplification -- it's a
misleading metaphor, and it represents a fundamental oversight of the
free culture movement.

``Speech'' and ``beer'' -- the choice of metaphors is telling. When we
compare free software to free speech, we cast it as a natural right
based on liberty, rather than a legal right based on property. This is
quite agreeable to US Americans,\footnote{I am from the United States.
  This essay is written from that limited perspective, and may not apply
  to other countries and cultures.} especially the techno-libertarian
set. We adore free speech, the most popular part of our first and
favorite amendment. Free beer, on the other hand, is a harder metaphor
to swallow. But the focus on speech, on liberty-based rights, does not
dispel the implications for property rights, only obscures them. Let's
take a closer look.

While the first and second freedoms in the Free Software Definition are
arguably matters of liberty, the third and fourth require the creator to
let users distribute copies, and modified copies, of their software. To
use a Free Culture license, as defined by Creative Commons, one must
similarly agree to allow adaptations of one's work for commercial
purposes. These licenses echo the demand of open scientists for access
to the experimental methods and results of other researchers, and the
insistence of music sharers and fanficcers in copying, modifying, and
remixing the media they love.

It's clear that developers, researchers, musicians and writers create
something of value. The free culture movement exhorts them to give that
value away. We say it's a matter of liberty, but mainstream culture
takes a different perspective, focusing instead on ``intellectual
property.'' Free culture advocates often reject the idea of intellectual
property, arguing that digital products, unlike food or cabinets or
cars, may be trivially copied. One can produce a thousand copies of
Emacs, or of Harry Potter, in a literal second. Without scarcity,
there's no need for property.

But scarcity is not a natural phenomenon, determined entirely by what is
technologically possible. Like so many things, it is socially
constructed. Humanity produces enough food to feed the world, enough
vaccine to wipe out a dozen diseases, and, in the United States at
least, enough housing to shelter our six hundred thousand homeless
brothers and sisters. Why should we direct our energies against
artificial scarcity in culture, when artificial scarcity elsewhere
causes more fundamental harm?

It's not surprising then that so many members of the free culture
movement are, like myself, immensely privileged. As the child of an
upper middle-class family, a United States citizen, a white, cis college
graduate, I have no fear that I will ever be hungry, homeless, or
without vital health care. Without persistent reminders of these
artificial scarcities, it is easy for me to focus on free culture; I can
ignore property because I have access to plenty of it.

Like many other free software activists, I have used the phrase ``free
as in speech, not free as in beer'' for years. But I have come to
understand that it is not an explanation but an equivocation. Free
culture absolutely has implications for property, and we need to face
them.

The schism between Free Software and Open Source Software can be
interpreted through the response to this problem. Free Software
advocates tend to embrace liberty rights, preferring not to think about
property, and often eschewing the idea of intellectual property
altogether (while retaining, for the most part, their belief in other
kinds of property). Open Source advocates, on the other hand, try to
reconcile the property implications of free software with the capitalist
culture in which most of it is produced. Open source, they argue, will
increases the value of your property. As Mako Hill notes in his essay
\emph{When Free Software Isn't Better},\footnote{https://mako.cc/writing/hill-when\_free\_software\_isnt\_better.html}
the Open Source Initiative's mission statement focuses on the higher
quality and lower cost of open source software. But, he continues,
free/open source software is sometimes of lower quality and lower value
to individuals and businesses. The reconciliation of free software and
capitalist culture, always fragile, falls apart.

But the open source approach is not the only way to come at ``free as in
beer''. The private capital of businesses using open source isn't the
only kind of property. There is -- and always has been - the commons.

It is easy to reframe the arguments for free culture around the commons.
The case for open science becomes the case for public knowledge. The
case for free distribution of art and literature becomes the case for
shared culture. And the case for free software becomes the case for
collectively built, collectively-evaluated technology. Free culture,
then, is a movement which advocates universal access to a common good.

This is not a new perspective, of course. One of the most well-known
free culture organizations, Creative Commons, uses precisely this
framing. But many others reject it, and even those who embrace a digital
commons often ignore the pressing threats to our natural and social
commons. They advocate for free culture but not for public education,
universal health care, guaranteed housing, and basic income, or their
equivalents

This is not just a matter of morality. The lack of a fiercely protected
natural and social commons endangers the digital one. In a scarcity
society, our labor must be hoarded jealously. People don't have time to
learn about their computers, submit patches to projects, seek out free
music instead of stolen music. They don't have the security to publish
in open access journals, to protest surveillance, to give away their art
or their software in hope of future reward. Many who would love to
participate in free culture cannot, as Ashe Dryden lays out eloquently
in her piece \emph{The Ethics of Unpaid Labor and the OSS
Community}.\footnote{http://www.ashedryden.com/blog/the-ethics-of-unpaid-labor-and-the-oss-community}
Like unpaid political and literary internships, free software
contributions act as a filter, allowing only the privileged to
participate.

It's tempting to wave away this last issue by arguing that less
privileged people have greater access to free culture than to
proprietary cultural products. After all, we're giving it away! But
accessibility is seldom a priority in free culture -- in free software,
many projects are made for other developers and we celebrate
``scratching your own itch''. Not that a focus on less privileged people
is always better -- in fact, it can be deeply condescending and
unhelpful. No, these arguments miss the point entirely. The groups
under-represented in free culture are not hamstrung primarily by lack of
access to the digital commons, but by threats to the natural and social
commons.

Acting in solidarity with the struggle for physical security and against
abuse is not only the right thing to do, it benefits all of us. When the
free culture movement represents the fullness of human diversity,
scratching your own itch will leave everyone satisfied. When it contains
everyone who shares its values, we'll have the resources and the reach
we need to ensure a vibrant and widely-treasured digital commons.

We live in alarming times. Even the computers with which we create these
digital gifts are made, too often, by people trapped in abusive
conditions, using processes that blight our primal Commons, the global
environment. We cannot abstract away these facts; we cannot advocate for
free culture as though in a vacuum. We must advocate for the commons in
all of its forms -- digital, social, economic, environmental -- before
the cost of freedom becomes too high to bear.
