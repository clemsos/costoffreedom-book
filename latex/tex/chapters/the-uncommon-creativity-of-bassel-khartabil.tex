\subsection{The Uncommon Creativity of Bassel
Khartabil}\label{the-uncommon-creativity-of-bassel-khartabil}

\begin{quote}
\hyperlink{barry-threw}{Barry Threw}
\end{quote}

\begin{quote}
The people who are in real danger never leave their countries. They are
in danger for a reason and for that they don't leave \#Syria

@\href{https://twitter.com/basselsafadi/status/164355948582932480}{basselsafadi}
on Twitter, 1/31/2012 14:34:46, one month before detention.
\end{quote}

In October 2010, I sat at a checkpoint on the Lebanon-Syria border,
waiting for Bassel. It was late, and I'd been sitting in a nearby café,
smelling of bleach but otherwise unremarkable, for nearly 12 hours. I
was waiting with one of my traveling companions, Christopher Adams, who
had been denied entry as a result of visa issues (``everything fine,
stamps just changed yesterday''). We were part of a group of Creative
Commons advocates traveling to Damascus as the last stop on a tour
around the Arab world, doing workshops on free culture and open source
software, along with such community stalwarts as Joi Ito, Lawrence
Lessig, Mitchell Baker, Jon Phillips, and Bassel himself. It was a group
from the near-future, time traveling at a second-per-second to the
oldest still-inhabited city in existence, a place outside of time.

It was clear, after much whispered negotiation between Bassel and the
border police, that Christopher wouldn't be admitted via one of the
usual persuasions employed to skirt the bureaucratic impasses typical
for that part of the world. Bassel spent several hours on his cell
phone, serially calling government offices of murky authority, but
eventually it became apparent that a resolution required in-person
meetings. Bassel and the majority of our crew left for Damascus, leaving
Chris and me to enjoy the landscape, a sepia liminal space of Martian
desert and cinder-block buildings where used washing machines and cell
phones were sold. As the night wore on and nothing changed Chris went
back to Beirut, and I sat while Bassel allegedly made his way back for
me.

These moments didn't stand out to me at the time, but it was here I
first was affected with great admiration and respect for Bassel
Khartabil, through watching his tireless commitment to his friends, and
later learning of his larger efforts enabling access to knowledge,
preserving cultural heritage, and fostering free creative expression.
The projects he's created and supported, the artifacts left behind,
reveal an astonishing intuition for issues holding back society in Syria
and globally, and a singular vision for building technical and social
ways to address them. Organizing this trip to Damascus for luminaries of
the open culture/free software movement was exemplary of what brings him
joy: bringing his friends and colleagues together, and sharing the
knowledge and experience of his home.

Bassel Khartabil was born in Syria in 1981 of a Palestinian father and
Syrian mother. Although born in a culture known for its conservatism and
adherence to tradition, he was raised as the only child in a liberal and
creative household; his father, Jamil, a writer, and his mother, Raya, a
piano professor. As with many only-children, Bassel was most at home
inside his own curiosity and creativity. An avid reader, he devoured
advanced books on the ancient history of the Middle East, and Greek
mythology, from a young age. He was also a natural self-learner and
taught himself English from a CD-ROM on his father's computer. He was
drawn to computers, helping his father research online, and learning to
program in C. This fascination and facility with technology continued
throughout his upbringing, fixing his family computers, learning
advanced programming for desktop and the web, and joining the
communities dedicated to advancing and upholding the openness and
creativity that he cherished. He was raised in a place of rich history
and tradition, but lives in a global world of technology; a man outside
of time.

Bassel, like many of us, found Freedom within technology, and tried to
share that freedom with others, but he did not yet know the cost.

If there is one thing always said about Bassel by the people that know
him best, it is that he loves to share is knowledge with anyone who
asks. For two weeks we lived out of
\href{https://wiki.hackerspaces.org/Aiki_lab}{AikiLab}, the
``hackerspace'' he founded in Damascus, giving workshops and lectures,
and meeting the young community that came to listen. The space was for
more than just events, it was a social gathering place, where knowledge
was shared, and new friends and collaborations made. Inside were
computers, projectors, the Internet, all of the equipment needed to
provide education and support to the nascent Syrian tech culture. But,
the vital element was not the gear or even AikiLab, but Bassel himself.
Even when he was confined in Adra prison, Bassel found time to teach the
other prisoners English and about technology, even though they had no
computers available.

But, even more than education, Bassel's true gift is Protoculture,
developing the near-future alpha versions of projects catalyzing change
in cultural contexts, whether software tools, community organization, or
digital art. His
\href{https://en.wikipedia.org/wiki/Aiki_Framework}{Aiki} web
development framework allowed multiple developers to work simultaneously
on a live web site, while maintaining security. It was used to build
still active open content projects such as the
\href{https://openclipart.org/}{Open Clip Art Library} and
\href{https://fontlibrary.org/}{Open Font Library}. His platforms,
whether physical, social, or digital enable new projects to spring up,
and the community to build on its self.

Perhaps none of Bassel's cultural prototypes were more prescient than
the work he started around 2005, with a group of archeologists and 3D
artists, to virtually reconstruct the ancient ruins of Palmyra. One of
the world's most important archaeological sites, Palmyra stood at the
crossroads of several civilizations, with Graeco-Roman architectural
styles melding with local traditions and Persian influences. Little
could Bassel know that ten years after he began, Daesh fundamentalists
would be actively deleting this architecture embodying Syrian, and the
world's, cultural heritage. But his foray into digital archaeology and
preservation created a time capsule that will be invaluable to the
public, researchers, and artists for years to come.

Tragically, Bassel has not yet been able to complete this project. On 15
March 2012, Bassel was imprisoned by the Assad government in a wave of
arrests triggered by the civic unrest pushing for democratic freedom in
Syria. The United Nations Working Group on Arbitrary Detention has
determined that Bassel's arrest and imprisonment were arbitrary and in
violation of international law, and has called for his immediate
release. For three years, he was held in the infamous Adra prison with
7,000 others, until October 2015, when he was moved to an unknown
location. As of this writing, no information has been released by the
Assad government on his location or condition. The
\href{http://freebassel.org/}{\#freebassel} campaign continues to fight
to keep Bassel's plight in the public eye, and, ultimately, achieve his
release. For Bassel, the Cost of Freedom has not been trivial or
abstract, but has caused him to be separated from his community and
loved ones.

We have recently launched a project building on Bassel's original work
called \href{http://newpalmyra.org/}{\#NEWPALMYRA}. It is an online
community platform and data repository dedicated to the capture,
preservation, sharing, and creative reuse of data about the ancient city
of Palmyra. Released under a Creative Commons CC0 license, all models
and data collected are available in the public domain to remix and
distribute. The project will continue, continued by its international
affiliates and advisors, until Bassel's release, when he can accept his
research
\href{http://joi.ito.com/weblog/2015/10/22/mit-media-lab-r.html}{position}
at the MIT Media Lab and carry it forward once again.

The \#NEWPALMYRA project starts from Bassel's original vision, but goes
further, creating a new community around the virtual Palmyra through
open calls for participation, real world development events, and pop-up
art shows. A city is built in architecture, but lived in by people, and
our virtual New Palmyra will serve as a nexus for creative explorations
and cultural understanding. The book you are reading is one of these
related projects, bringing together writings from a diverse and
insightful group of authors committed to the promise of free culture.
Here we create our own time capsule, a record of thoughts on freedom and
responsibility from many different perspectives and disciplines, so the
next generation of digital archaeologists can learn about us.

Eventually, Bassel came walking through the dark to that checkpoint, and
with more whispers to lackluster guards I was on my way to Damascus.
Christopher met our group the next day, and together we all embraced
Bassel's world, one of standing up for freedom, and constantly giving to
his friends and community, that to this day inspires us to push further.
This Uncommon Creativity, an ability to innovate and invent in the
future while building on the past, is what makes him a vital visionary
for the Syrian community. But, I find myself once again waiting for
Bassel, this time to regain his Freedom for which he has paid so dearly.

I hope, my friend, to see you soon.
