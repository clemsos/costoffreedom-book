\subsection{The Burden of Journalism}\label{the-burden-of-journalism}

\begin{quote}
\hyperlink{theophile-pillault}{Théophile
Pillault}
\end{quote}

An Infinite and Unsolvable Debt

We practice journalism as we are in an age of working. However, after
more than 15 years of reports and interviews, we are still not able to
call it a job, because its cost has been so high in comparison with the
rewards. High for our lives, precarious and submitted to media that
don't even deserve our attention. High for the profession itself, which
we happen to sometimes soil with doubtful deontological hygiene, or even
worse, mimeticism.

Dealing with a less and less united journalism practice and even
definition, reporters and information collectors are getting more and
more individualistic. Journalism has always been a game for rich people,
as many of them know, but it's getting dangerously worse as the job
becomes more precarious, leading to an economic reign of division.

In France, the number of syndicated journalists is totally meaningless
as they are thrown into a profession ruled by the publication race
contest. For instance, less than a quarter of French registered
photojournalists are members of a professional organization.

Journalism doesn't have the time any more to think itself through:
reports follow each other at a rhythm the reader can't keep up with. And
it doesn't really matter as they are all the same.

It seems that journalistic narration has been reduced to reporting on
instabilities such as conflict geographies, financial markets mobility
or multinational successfulness. For a few years, the Syrian battlefront
seems to have become the only place for photojournalists to do their
job. In France, mass-media contribution is limited to vox pops about
arriving refugees or Greek debt. When it is not busy exploring gossip
magazines, a massive part of specialized media just settles for
streaming Facebook or Google citations.

In those conditions, it is hard to draw attention to celebrated Internet
volunteer Bassel Khartabil, an open web developer who has been
wrongfully detained in Syria since 15 March 2012. This prisoner stands
at complicated crossroads between media and international stakes, and
talking about his disappearance implies levels of analysis that the
French media, it seems -- if only they were the only ones -- cannot
handle. This fight for information, amongst other fights, struggles to
make its way into newsrooms which look more and more like dominant
system backrooms, left alone by any form of resistance.

As we face those ideological barriers, how can we hope to get more
people outside those little circles already convinced to read dissident
analyses?

Today, we still haven't found a satisfying setup to provide for the
production and diffusion of chemically pure information, purified from
political, institutional or personal stakes.

The same questions apply to online journalism. Neo-data journalists?
Webdoc producers? Datavisualizers? They are under the same pressures as
their paper ancestors. The Internet can't produce another type of mass
information free from the rentability logics and industrial
concentration that strikes the sector.

Sharing our analysis, dissidence or images in free information
frameworks provides last victories for the small media people. But for
how long? After 15 years of articles written on the edge, of unpaid
reports, of lots of often spoiled written material, isn't it time to
listen to reason and look for evidence? What would they say?

Maybe that this job doesn't exist or doesn't exist anymore.

However, very few societies can pretend to emancipate themselves without
a free information system. So, we have to stand at the frontlines.

Because there is no cost for journalism or ideas, Bassel Khartabil is
detained today, and other women and men will be. There is no cost of
freedom. Just an infinite and unsolvable debt that nothing can resolve.

Let's honor this debt, until our specificities eventually start to
resonate, from newsrooms to media schools and beyond that, in all
societies willing to free themselves.
