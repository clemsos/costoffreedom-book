\subsection{Free Software Economics}\label{free-software-economics}

\begin{quote}
\hyperlink{hellekin}{hellekin},
\hyperlink{jaromil}{Jaromil},
\hyperlink{radium}{radium}, and
\hyperlink{christian-grothoff}{Christian
Grothoff}
\end{quote}

Fifteen years ago, in his seminal article
\href{http://harvardmagazine.com/2000/01/code-is-law-html}{Code Is Law},
Lawrence Lessig identified a problematic: \emph{``The most important
contexts of regulation in the future will affect Internet commerce:
where the architecture does not enable secure transactions.''} Today,
European free software researchers are implementing innovative solutions
to address this and other issues that will shape digital economics in
the near future.

We argue that beyond regulation, code embeds politics. We'll introduce
two projects we think will transform not only how we conduct economic
transactions online, but which also hold the potential to radically
change the global balance of economic power.

\emph{Freecoin} is a social digital currency based on the blockchain
technology of Bitcoin but which relies on a ``social proof of work''
instead of the original brute-force algorithmic proof of work used in
Bitcoin. Freecoin was developed by the Dyne Foundation, a free culture
foundry based in the Netherlands, and now a European Research Network.
Freecoin is Project no. 610349 in the FP7 -- CAPS framework, under the
Decentralised Citizens ENgagement Technologies (D-CENT) project.

\emph{GNU Taler} is the Taxable Anonymous Libre Economic Reserve, a new
electronic payment system under development at Inria, the French
National Institute for Information and Automation Research, and the
Technical University of Munich (TUM). It aims at delivering an online
and offline payment solution for various established currencies such as
Euro, U.S. Dollar, or even electronic currencies such as Freecoin.

Together they implement a unique electronic solution for mainstream
economics beyond payment. They were specifically designed with social
values addressing the shortcomings of both early electronic currencies
such as Bitcoin, enabling a variety of local currencies to work
together, extending transactions to non-monetary domains such as
distributed storage, and drastically limiting the criminal use of money.
Their combined approaches unfold a many-to-many platform suitable for
daily use from global micro-payments to local social currencies.

Bitcoin was the first digital currency to appear on the Internet. It
implements a distributed and authenticated public ledger called the
blockchain, whose mode of operation is based on decentralized consensus.
The blockchain replaces the bank: it uses cryptographic techniques to
regulate the emission of coins and verify transactions between peers.

The design of Bitcoin has definitive shortcomings: first of all it's
very volatile. By the time this article was finished, its value was down
to USD 402.7 after reaching USD 479 earlier during the day. As all
finalized Bitcoin transactions appear in the blockchain, the whole
market is transparent, and a coin's history can be used to connect
identities to addresses. To avoid double spending, no bitcoin
transaction can be reversed, which means the buyer is not protected
against fraud from the seller, nor addressing errors. By design, Bitcoin
rewards early adopters. Finally, the proof of work requires a
significant amount of computing power which translates into high energy
costs.

\subsection{Freecoin}\label{freecoin}

\href{http://freecoin.ch/}{Freecoin} is a set of tools that let people
run a reward scheme that is transparent and auditable by other
organizations. Designed for participatory and democratic organizations
willing to incentivize participation it is, unlike centralized banking
databases, a social currency that is reliable, simple, and resilient.
Technical and design elements shape a way to legitimize the bottom-up
process using audit of cryptographic blockchain technologies such as
decentralized storage, ubiquitous wallets, and ad-hoc social
remuneration systems.

The Freecoin project insists on the need to strengthen the democratic
debate necessary to consolidate and preserve the management of economic
transactions, especially those with a social orientation, inside the
local monetary circuit. It focuses on complementary currency design to
allocate and distribute credit created among engaged members, using a
reputation as risk management system.

Citizens can collectively define their social needs using a
participatory deliberation based on ``social sustainability'': without
participation, local monetary circuits run the risk to remain too
little, too dependent on the local political cycles, too far from the
real demand that may be expressed by the local economic system. Choices
need to be informed with social objectives and ethical criteria to
properly allocate resources and investments.

The Freecoin / D-CENT project is an experiment in digital social
currency design that aims at solving two problems: (1) the vulnerability
of centralized information systems, whose integrity can be jeopardized
by compromising a few points of failure, and (2) the management of
digitally distributed trust to make sure that different organizations
which may not share trust can agree and verify the integrity of a
transaction history, even in the absence of the other organization.

1 ) \emph{Complementary currency governance systems}: with a
minimalistic reinterpretation of the blockchain technology, the Freecoin
Toolchain is a toolkit for community members to easily access and decide
on the features of their currency system by using a decentralized
governance structure -- essentially, bringing back human intervention to
oppose the high-frequency trading algorithms (Durbin, 2010). A system
for collective deliberation on the decisions regarding digital currency
will allow users to engage in collective monetary policy-making.

2 ) \emph{Distributed trust management systems}: reputation is the basis
for trust and decision-making. Putting together trust and the
blockchain, the Freecoin Toolchain allows for the design and prototyping
of systems aimed at managing social currency in a community,
i.e.~reputation in a decentralized fashion. The use of
micro-endorsements allows the even spreading of risk among participants,
and the rewarding of the best political contributions (similar to the
participatory budgeting in Iceland). In a municipality, the use of those
credits as loyalty scheme vouchers lowers the risk to promote proposals
that go against the common interest of the citizenry.

The issuance of new coins is a technology-driven mechanism based on a
consensus algorithm that neutralizes counterfeiting. However, this may
also be seen as a departure from an active and critical engagement among
humans and machines, whereby the creation of money in the system is
motivated by social interactions for the common good, rather than by
exclusively hashing cycles and shortsighted money-making. Therefore, the
task of the Freecoin / D-CENT research is to redefine Bitcoin's `proof
of work' and the reward of a blockchain system, to devolve power into
the hands of people through a democratic decision process. The outcome
of this shift in design is twofold: (1) people engage in transactions
that have real world desirable impact that they produce and collectively
construct; (2) new participants can enjoy an egalitarian economic
environment by avoiding the undesirable condition of structural
advantage by early adopters of a currency. At the same time, this allows
complete democratic oversight of transaction history and collective
deliberation on social currency system rules of engagement and reward.

The Freecoin project is licensed as Affero GNU General Public License
version 3 or later to make sure that all uses, commercial or
non-commercial, will provide access to the source code, be it modified
or not.

\subsection{GNU Taler}\label{gnu-taler}

At IETF 93, Edward Snowden said via videoconference: \emph{``I think one
of the big things that we need to do, is we need to get away from
true-name payments on the Internet. The credit card payment system is
one of the worst things that happened for the user, in terms of being
able to divorce their access from their identity.''} So while obviously
some people do not care much about their privacy, we do think that many
will heed his words once a viable alternative exists. Identity theft,
fraud, convenience and efficiency gains are other reasons why consumers
or merchants are likely to be excited about adopting Taler.

While our initial market is likely to be technological enthusiasts with
a focus on privacy, we believe that the technology is applicable in
general for all payments (in online stores and physical stores) assuming
sufficient engineering effort (integration, ease of use, etc.) is put
behind it.

However, as the receivers of funds are not anonymous and can be audited
and taxed by the state, Taler's market does not include tax evasion,
money laundering, human trafficking and any other forms of illegal trade
that have ballooned the popularity of Bitcoin.

Established payment systems, such as the ubiquitous credit cards, try to
authenticate the user making the payment. In contrast, Taler uses
cryptography to secure the value and validity of the payment. As a
result, identity theft is no longer a problem for customers using Taler,
and merchants also do not have to worry about the theft of sensitive
customer information. Naturally, customers may reveal their identity
(i.e.~for shipping), but they are not forced to by the payment system.
In contrast to previous research designs, Taler provides stronger
assurances for the customer's privacy (including better than BitCoin,
where transactions are linkable). We are also the first electronic
payment system of this type that supports giving change (i.e.~pay 5 EUR
with a 100 EUR coin and get 95 EUR in electronic change) with these
privacy assurances. Taler can even provide refunds to customers without
violating their anonymity. At the same time, transaction costs are
several orders of magnitude cheaper than those with
BitCoin-technologies. At scale, we expect transaction costs to be lower
than those for existing credit cards, as expenses from fraud by
consumers, merchants or identity theft are prevented by the
cryptographic protocol.

Unlike BitCoin, Taler does not introduce a new currency but merely
provides digital representations of existing currencies (such as EUR,
USD or even BTC), eliminating the risk from currency fluctuations
introduced by payment systems that introduce a new currency, such as
BitCoin, AltCoins, or Stellar.

Our system consists of various components operated by different groups.
The mint creating the digital coins is mostly finished and just
undergoing additional testing and audits. The mint is also the most
complex part of the design. Even after this is finished, we still need
to integrate the mint with the banking system of each respective country
to perform wire transfers. This is a one-time expense per banking
system. For the customers, we need to ensure that the ``wallet''
application works well for their respective platform. Our initial
implementation is for Firefox, ports to other browsers and native apps
for mobile phones will require more work. The wallet is simpler than the
mint, but still non-trivial especially if we want to make it easy to use
and nice to look at.

Finally, each merchant will require some modifications to their business
logic to integrate the new payment system. While these modifications are
way smaller and easier than the mint or the wallet, there are of course
many more businesses platforms than browsers or banking systems. Hence,
while the work for an individual store should be tiny, this will be a
major effort. We are trying to document our protocol and prototypes and
will provide reference implementations in various languages to
facilitate this integration.

GNU Taler is free software released under the terms of the GNU General
Public License version 3 or later.
