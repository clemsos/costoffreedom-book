\subsection{The Cost of Internet
Freedom}\label{the-cost-of-internet-freedom}

\begin{quote}
\hyperlink{geert-lovink}{Geert Lovink}
\end{quote}

Dedicated to Bassel Khartabil, written for the Cost of Freedom Book
Sprint.

Every act of rebellion expresses a nostalgia for innocence and an appeal
to the essence of being.'' -- Albert Camus

Let's translate Isaiah Berlin's ``Two Concepts of Liberty'' from 1958 to
our age. Berlin distinguishes between negative and positive freedom:
there is the negative goal of warding off interference, and the positive
sense of the individual being his or her own master. In both cases, a
fundamental distinction is made between the autonomy of the subject and
the crushing reality of repressive systems. For Berlin, freedom is
situated outside of the system. Written in the shadow of
totalitarianism, at the height of the Cold War, there wasn't much else
for him to expect. In that period, the notion of freedom as an everyday
experience was absent. The existentialist gestures after World War II
emphasized the legal rights of the individual-as-rebel who stood up
against evil outside forces.

Right at the beginning of his famous essay, Berlin formulates Evgeny
Morozov-type sentences that sound remarkably familiar to those involved
in contemporary `net criticism' debates.

``Where ends are agreed, the only questions left are those of means, and
these are not political but technical, that is to say, capable of being
settled by experts or machines, like arguments between engineers or
doctors.'' And he continues: ``That is why those who put their faith in
some immense transforming phenomenon must believe that all political and
moral problems can be turned into technological ones.''

Berlin reminds us of the phrase of Friedrich Engels about ``replacing
the government of persons by the administration of things.'' Sounds very
timely, no? But wait, is this an old communist phrase, or a libertarian
dogma preached by Silicon Valley billionaires?

Fast forward ten or twenty years and the concept of `the system' is no
longer perceived as alien. In the 1970s, the idea spread that (computer)
systems were man-made and could be programmed, designed, and thus
democratized. The critique of the technocratic society that we can trace
in the memories of Albert Speer, published in 1969, were soon to be
forgotten and taken over by a fascination for the do-it-yourself spirit
of the garage hackers. Instead of looking at IBM mainframe computers as
a tool of 1984's Big Brother, the personal computer was introduced as a
portable counter-cultural alternative, intended to undermine power as
such and break it up into a 1001 fragments of decentralized, distributed
expressions of human creativity.

Jump another thirty years onwards, and Internet freedom activists run up
against very clear boundaries and setbacks. Liberal obsessions with
privacy and copyright are still interesting but no longer essential in
order to understand the big picture. What's at stake is much larger than
a bunch of legal issues, defined by lawyers. What's necessary is a
comprehensive understanding of the political economy of the Net,
combined with critical knowledge of global politics. The legal
strategies have run empty. It is now all about power politics and
organization of the field. The loose ties that social media have left us
with do not foster long-term collaborations but force us into a 24/7
cult of the update.

The philosophical question, can we find freedom inside the machine,
should be answered with a definite no. So far, programmers, geeks and
artists have stressed the possibilities of carving out small pockets for
themselves, in order to realize their free software and creative commons
projects. This `temporary autonomous zones' approach has a liberal
consensus as its premise, that the `Internet' will tolerate such
experiments within its infrastructure.

The original Internet freedom within the system is shrinking as we
speak, and we lack the appropriate tools and strategies to do something
to counter it. Soon we will be back at square one, demanding freedom of
the Internet.

The ideal of freedom outside of the Matrix will not necessarily be
Luddite in nature. The coming uprising against the Internet as a tool of
surveillance and repression will be technologically informed, and needs
to be distinguished from the related human right to have time off work
and have a life. This ain't no offline romanticism. Our memes need to
communicate this simple message: positive Internet freedom is the road
to serfdom. We need to revolt against the soulless, mechanical ideas of
the Silicon Valley engineering class and their solutionist marketing
slogans. In order to prepare ourselves, we need an understanding of the
Two Concepts of Internet Liberty.
