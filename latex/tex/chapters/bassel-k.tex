\subsection{Bassel K}\label{bassel-k}

\begin{quote}
\href{../appendix/attributions.html\#marc-weidenbaum}{Marc Weidenbaum}
\end{quote}

I read ``The Trial'' at too young an age. It instilled in me many
things, some of them even positive, such as an affection for Franz
Kafka, an aspiration to taut structure, and a desire to tell stories. It
also haunted me, and it does to this day. It imprinted on me an intense
fear of undeserved imprisonment.

I was introduced to the imprisonment of Bassel Khartabil by three
remarkable people: Niki Korth, Jon Phillips, and Barry Threw. They are
in many admirable ways as free as Bassel is not. Each of the trio is
dedicated to their own individual and collective artistic pursuits to
explore the deep potential where technology and culture meet. They make
and celebrate the things that make today a special time.

And they know full well that all is not right in our time. They expend
significant energy in building awareness of the ongoing fact of Bassel's
murky, tragic legal status. At their suggestion, back in January 2014, I
gathered musicians to highlight Bassel's plight. These musicians
participate collectively in something called the Disquiet Junto. It's a
freeform group I moderate that each Thursday responds to
music-composition prompts. The idea behind all the prompts is that
creative constraints, such as those employed in Oulipo and Fluxus, are a
useful springboard for creativity and productivity.

The Junto's fondness for such ``constraints'' met a fierce complement
when we tackled Bassel's situation, which is that of a most uncreative
form of constraint. There were many ways we could have paid tribute to
Bassel. What we elected to do in the Junto was to keep one of his
projects going: he may be in jail, but his art could continue to
develop. Prior to Bassel's arrest on March 15, 2012, in Damascus, he was
working on several projects. Among them was a three-dimensional computer
rendering of the ancient city of Palmyra. What we in the Junto did was
make ``fake field recordings,'' audio of what the halls of Palmyra's
structures might have sounded like millennia ago. Much as Bassel was
trying to revive an ancient world, the Junto participants were, in
essence, keeping one of his projects alive while he is incapable of
doing so. And, of course, building upon his artistic efforts was true to
the ethos of the Creative Commons, in which Bassel has been profoundly
engaged.

We had no idea, of course, back in early 2014, that Palymra would itself
receive worldwide attention when ISIS, the extremist movement, would in
2015 move to destroy much of the ancient city's remaining architectural
history, or that, later still, Russian warplanes would further damage
the site. This is one of Kafka's lasting legacies: just when things seem
horrible, they can and do get worse.

Palmyra has fallen. Bassel remains in jail. The challenge to rectify his
situation has long since surpassed the overly employed term
``Kafkaesque.'' Someone must have been telling lies about Bassel K,
because he is still kept from his freedom. But as long as he is in
prison, there are plenty of people telling his story, and keeping his
work alive.
