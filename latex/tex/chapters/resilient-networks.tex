\subsection{Resilient Networks}\label{resilient-networks}

\begin{quote}
\href{../appendix/attributions.html\#jean-noel-montagne}{Jean-Noel
Montagné}
\end{quote}

\textbf{Makers}

In France, most of the Makers are hobbyists, technolovers, geeks that
create for fun, for local glory and some, for business. But few of them
are makers for social or political goals. Most of the objects created in
fablabs and makerspaces in the last years are useless regarding the
urgent problems of the planet. Because the planet is on fire. Climate
crisis. Energy crisis. Demography crisis. Water crisis. Metals crisis.
Financial crisis. Education crisis, even crisis of mental health because
of the abuse of digital communication.

But stop ! it's enough ! Come back to transformaking, No-one wants to
hear about this crude reality !

\textbf{Resilient Society}

And that's the problem: historians studying the extinction of old
civilisations in the last millennia have discovered that leaders and
populations knew about the perfectly serious problems of their time, but
they ignored the scientific advice and all indicators turning to red,
until the end. We are doing exactly the same and we don't have a lot of
time to act. We must transform all sectors of the society before the
conjunction of some important crisis, and transformakers will help us to
do it.

In the global village, industry is totally dependent on flux, networks
of raw materials, energy, goods, tools, components, distribution and
transportation. Any failure in one spot can disturb or stop the whole
chain, from extraction of raw materials to distribution of goods. This
interdependency is an enormous fragility in the context of the coming
crisis, and transformakers can help us to break it.

We have all noticed that we can't really count on our political systems
to find efficient solutions. We know we can only count on ourselves. We,
citizens, can build resilient communities, based on small structures,
driven by direct democracy, and based on big citizenship networking. We
have the digital tools and the network to do it.

Transformakers have a key role in this transition from globalisation to
resilience.

\textbf{Global Crisis}

The COP21 UN conference about climate change offers to limit the rise of
the global temperature to two degrees more. Accepting 2 degrees more, on
average, for the planet, however, is accepting violent transformations
of the climate that will create a giant loss of biodiversity, massive
extinction of species in earth and ocean during those years. Two degrees
more will also create huge environmental and social disorders,
instability everywhere, wars, starvation. Hundreds of millions of
refugees will have to move, dismantling completely the actual geopolitic
equilibrium.

Pure water is also missing everywhere because of very bad management,
but the most important resource crisis will come with metals and oil. We
live now with the illusion of infinite resources, but this new
prosperity will have an end. The planet has a limited quantity of fossil
energy in the ground, and we are reaching the limits in one or two
generations, in our children's lifetimes. No lessons have been taken
from the 2008 crash. Improvements in high-frequency trading do not
actually cover the many debts of countries and their people. Big
monetary regulators are also provoking also a crisis of democracy, of
citizenship, of trust in each other.

A new era of chaos brings an opportunity for radically changing the
system in good directions.

\textbf{Transformaking}

How can transformakers help in the context of global crisis ?

By helping us to change the scale from globalization to small resilient
networked communities, to rebuild real direct democracy and redefine
urbanisation and the usage of our lands. By helping us to rebuild our
social organisation around knowledge networks. By helping us to harvest
clean energy, renewable energy everywhere, and to share it. By helping
us to redefine our material strategies, our industrial strategies. By
creating new models for currencies and money circulation.

We discover today that good social, environmental and financial
practices have always existed. Transformaking is the common behaviour in
many communities in the world, especially in rural areas: do it
yourself, DIWithOthers, Do It Together: people invent tools and
technologies adapted to their context, to their pragmatic needs, using
few resources, using local resources. People repair, they recycle, they
hack objects, they transmit the knowledge to young generations. Poor
countries will not suffer as much as rich countries in the chaotic
future, because they have always lived in the Transformaking way.

In social organisation, all over the world, small communities use
solidarity structures, monetary arrangements, like barter systems that
can be considered as local money, in a pure peer-to-peer exchange. The
organisation of traditional communities offers big lessons for us and
this model just needs digital tools to be adapted to small communities
in the modern world.

\textbf{Sharing knowledge = open sourcing}

Transformaking officially arrived in our society 30 years ago, when
hackers started to change the world with the first open source software
licences, one of the most powerful political acts of the XXth century.
Artists followed the movement 20 years ago with open source documents
and artwork licences, and some makers have taken another important step,
ten years ago, with Open Source Hardware licences. This is
transformaking: changing the society by offering alternatives containing
the values of solidarity and knowledge.

Open source technologies, from their concept of production and
distribution, open the possibility of a total citizen control on
technology. It's now possible to envision human-scale industry, citizen
industry, decentralized industry, like our ancestors did before the
Industrial Revolution. The ecosystem of transformaking is self-organised
around knowledge networks. Any technological process can be created or
improved by transformakers, because networks of knowledge, networks of
citizen research, networks of materials and networks of components
exists underground. In the recent years, transformakers have started to
design and build very complicated open source machines related to many
sectors of industry, and citizen research now attacks topics such as
high tech medicine, nuclear physics, nanotechnologies or genetics. All
in Open Source: Free Libre Open Source Software (FLOSS) and Free Libre
Open Source Hardware (FLOSH).

Patents are living their last twenty years, even in some very protected
niche industries, such as medical equipment: look at their websites and
initiatives.

One could argue that hackers, transformakers are not regulated by
authorities, certifications, ethics commitees and could launch projects
which are dangerous projects for society. But no. They wouldn't. Because
transformakers are a network of citizens, we are self-organized and the
debate is always open in open source technologies. Creation and
correction of code, of designs, follows real democratic rules, much less
dangerous than government or military-security projects. How to promote
more transformaking in society?

First by protecting the Internet and net neutrality. Networking tools
are essential for democracy and sharing of knowledge. Big companies like
Facebook and GAFAM are silently killing the Internet by replacing all
software on the client side, by services driven by their data-sucker
servers, associated to the Panopticon of the Internet Of Things. New
global totalitarianism.

We can promote transformaking

\begin{itemize}
\tightlist
\item
  by supporting hackers and transformaker projects through crowdfunding
\item
  by opening new medialabs, hackerspaces, makerspaces, and open
  laboratories, and specializing in them ( biology, health, agriculture,
  etc)
\item
  by opening places in cities to dismantle, repair, recycle objects,
  parts, etc
\item
  by choosing to use open source software and open source hardware when
  available
\item
  by funding P2P and common goods initiatives in all sectors of society
\item
  by installing education programmes about hacking, about transformaking
\item
  by choosing slow and resilient communication technologies for
  establishing strong communication and education networks.
\end{itemize}

\textbf{Post capitalist era}

Transition from globalisation to new resilient small-scale networked
societies is necessary and must start now. Transformakers are the first
explorers of the post-capitalist era. But they don't move alone. Many
new citizen organisations, new-style political movements are following
the movement, but most of them ignore what transformakers are doing.

Transformakers have started to transform the society through new
behaviours based on local resources, local solidarities, self-management
and direct democracy, and based on global communication and global
exchange of knowledge.

Instead of losing energy to promote this vision into standard political
systems, we need to start building initiatives around us, responding to
our values, co-existing with the actual system, and if our alternatives
are good, if our models take sense into the society, they will naturally
replace the old system, without war, without revolutions.

\textbf{Let's do it. DIY, DIT, DIWO, DIN}\footnote{Do-It-Yourself,
  Do-It-Together, Do-It-With-Others, Do-It-Now}
