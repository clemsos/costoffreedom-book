\subsection{Why I Choose Copyright}\label{why-i-choose-copyright}

\begin{quote}
\hyperlink{lucas-gonze}{Lucas Gonze}
\end{quote}

I used to have a peculiar habit: I went to great lengths to not infringe
copyright. This was often misunderstood to be a statement in support of
stronger copyright, taking Metallica's side against Napster.

My intention was different.

Engaging with a cultural product increases its value, regardless of
whether the engagement produces immediate revenue. If you watch a movie,
you help to give it cultural currency, meaning the kind of thing that is
referenced in conversation.

If you watch a hit TV show and then talk about it, you make other people
want it. If you sample it, you make other people want it.

This is regardless of whether infringement is involved. If you are never
going to buy something, there is no loss of revenue when you don't pay.
If no revenue is lost, then the holders of the copyright have benefited.

Why would I have to infringe to access the work? There might be a
literal price (e.g. \$20 for a CD) that was too high. Or the work might
only be available on terms that I can't accept. For example, there might
be DRM, or I might need a cable TV account. Those terms are a form of
cost.

If I couldn't accept the price, and then infringement led me to help
increase its value, I wouldn't be helping myself. If I refuse to engage
at all,then I maximize the pressure I can exert on the vendor.

The vendor's ideal outcome is for me to pay the asking price. But the
second best outcome, if I can't do that, is for me to help convince
others to pay the asking price. The worst outcome is if I ignore the
product.

I ignored products to create pressure on vendors to offer them on
acceptable terms.

Purism was necessary. No doing what the copyright owner didn't want,
even if I disagreed. No knowing infringement, no matter how absurd the
implications.

No torrenting, Linux ISOs aside. No stream ripping. No DJ sets on
Soundcloud. No singing Happy Birthday without a license.

Because by obeying the rules I could demonstrate why the rules need
changing.

Evangelism didn't interest me, though. No preaching, no seeking
converts. I just lived my life according to a dogma with only one
adherent.

A more committed missionary would have done it differently because I
could not have an impact this way.

What I was doing was a boycott. Boycotts rely on broad participation.

The people are not dogmatic, and they want torrents.

Over time, the vendors have gotten somewhat better. You can buy music
without DRM. You can buy HBO a la carte, as HBO Now, without having to
buy cable TV.

That was caused by market pressure. The masses are not purist, but they
do prefer reasonable terms to ugly ones.

At the same time, my standards fell. The rise of mobile caused the
computing experience to became so unfair, so centralized, so tightly
controlled that my expectations with regard to media seem comically
unrealistic. There's no chance of jail-broken phones becoming the stock
experience.

When computing users have so much less power, holding out for more makes
no sense.

Eventually, I softened my position to a more common one - pragmatism. I
now avoid infringement, but will sometimes do it if the alternative is
ridiculous. I now avoid ridiculous problems rather than seeking them
out. It's a big change. It means that when the cost of media is too
high, I will do my best to pay up anyway.
