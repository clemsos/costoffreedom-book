\subsection{The Open World}\label{the-open-world}

\begin{quote}
\href{../appendix/attributions.html\#lorna-campbell}{Lorna Campbell}
\end{quote}

In \emph{Open is not a License}\footnote{Hyde, A., (2015), Open is not a
  license, \url{http://www.adamhyde.net/open-is-not-a-license/}} Adam
Hyde has described openness as `a set of values by which you
live\ldots{}a way of life, or perhaps a way of growing, an often painful
path where we challenge our own value system against itself.'

To my mind, openness is also contradictory. I don't mean contradictory
in terms of the polar dichotomy of open vs.~closed, or the endless
debates that seek to define the semantics of open. I mean contradictory
on a more personal level; openness raises contradictions within
ourselves. Openness can lead us to question our position in the world;
our position in relation to real and perceived boundaries imposed from
without and carefully constructed from within.

In one way or another I have worked in the open education space for a
decade now. I have contributed to open standards, created open
educational resources, developed open policy, written open
books,\footnote{Thomas, A., Campbell, L.M., Barker, P., and Hawksey, M.,
  (2012), Into the Wild -- Technology for open educational resources,
  \url{http://publications.cetis.org.uk/2012/601}} participated in open
knowledge initiatives, facilitated open events, I endeavour to be an
`open practitioner', I run a blog called Open World.\footnote{Open
  World, \url{https://lornamcampbell.wordpress.com/}} However, I am not
by nature a very open person; my inclination is always to remain closed.
I have had to learn openness and I'm not sure I'm very good at it yet.
It's a continual learning experience. Openness is a process that
requires practice and perseverance. (Though sometimes circumstances
leave us with little choice, sometimes it's open or nothing.)

And of course, there is a cost; openness requires a little courage. When
we step, or are pushed, outside our boundaries and institutions, it's
easy to feel disoriented and insecure. The open world can be a
challenging and unsettling place and it's easy to understand the impulse
to withdraw, to seek the security of the familiar.

When large scale open education funding programmes first started to
appear, (what an impossible luxury that seems like now), they were met
with more than a little scepticism. When a major OER funding initiative
was launched in the UK in 2009,\footnote{UKOER,
  \url{https://www.jisc.ac.uk/rd/projects/open-education}} the initial
response was incredulity.\footnote{Campbell, L.M., (2009), OER Programme
  Myths,
  \url{http://blogs.cetis.org.uk/lmc/2009/05/20/oer-programme-myths/}}
Surely projects weren't expected to share their resource with everyone?
Surely UK Higher Education resources should only be shared with other UK
Higher Education institutions? It took patience and persistence to
convince colleagues that yes, open really did mean open, open for
everyone everywhere, not just open for a select few. One perceptive
colleague at the time described this attitude as `the agoraphobia of
openness'.\footnote{I cannot remember who said this, but the comment has
  always stayed with me.}

Although open licences and open educational resources are more familiar
concepts now, there is still a degree of reticence. An undercurrent of
anxiety persists that discourages us from sharing our educational
resources, and reusing resources shared by others. There is a fear that
by opening up our resources and our practice, we will also open
ourselves up to criticism, that we will be judged and found wanting.
Imposter syndrome is a real thing; even experienced teachers may fail to
recognise their own work as being genuinely innovative and creative. At
the same time, openness can invoke a fear of loss; loss of control, loss
of agency, and in some cases even loss of livelihood. Viewed through
this lens, the distinction between openness and exposure blurs.

But despite these costs and contradictions, I do believe there is
inherently personal and public value in openness. I believe there is
huge creative potential in openness and I believe we have a moral and
ethical responsibility to open access to publicly funded educational
resources. Yes, there are costs, but they are far outweighed by the
benefits of open. Open education practice and open educational resources
have the potential to expand access to education, widen participation,
and create new opportunities while at the same time supporting social
inclusion, and creating a culture of collaboration and sharing. There
are other more intangible, though no less important, benefits of open.
Focusing on simple cost-benefit analysis models neglects the creative,
fun and serendipitous aspects of openness and, ultimately, this is what
keeps us learning.

In the domain of knowledge representation, the Open World Assumption
`codifies the informal notion that in general no single agent or
observer has complete knowledge'\footnote{Open World Assumption,
  \url{https://en.wikipedia.org/wiki/Open-world_assumption}}. It's a
useful assumption to bear in mind; our knowledge will never be complete,
what better motivation to keep learning? But the Open World of my blog
title doesn't come from the domain of knowledge representation; it comes
from the Scottish poet Kenneth White\footnote{White, K., (2003), Open
  World. The Collected Poems, 1960 -- 2000, Polygon.}, Chair of 20th
Century Poetics at Paris-Sorbonne, 1983-1996, and a writer for whom
openness is an enduring and inspiring theme. White is also the founder
of the International Institute of Geopoetics\footnote{International
  Institute of Geopoetics, \url{http://institut-geopoetique.org/en}},
which is `concerned, fundamentally, with a relationship to the earth and
with the opening of a world'\footnote{White, K., (2004), Geopoetics:
  place, culture, world, Alba.}. In the words of White:

\begin{quote}
no art can touch it; the mind can only try to become attuned to it to
become quiet, and space itself out, to become open and still,
unworlded\footnote{White, K., (2004), `A High Blue Day on Scalpay' in
  Open World. The Collected Poems, 1960 -- 2000, Polygon.}
\end{quote}
